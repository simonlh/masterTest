\documentclass[11pt, norsk, a4paper]{article}
\usepackage{babel,graphicx,textcomp,varioref}
\usepackage[dvipsnames]{xcolor}

%farger jeg skal ha med
\definecolor{myYellow}{RGB}{241, 196, 15}
\definecolor{myBlue}{RGB}{52, 152, 219}
\definecolor{myGreen}{RGB}{46, 204, 113}
\definecolor{myGrey}{RGB}{149, 165, 166}

\usepackage[utf8]{inputenc}
\author{Simon Lysne Hyenes}
\title{Transcription of workshop 1}

\tolerance = 5000
\hbadness = \tolerance
\pretolerance = 2000
\begin{document}
\maketitle{}

\section{Introduction}

\section{Group 1}
This is the transcript from the second workshop. It lasted about 20 minutes and had 2 participants, hereafter called p1 and p2. In the first part the participant are looking at a prototype on a blackboard. The prototype consists of several images that are explained to the participants. 

\textcolor{myBlue} {Interviewer:} Hei, da har vi startet vi kan egentlig bare å begynne med å se på appen. Hvis dere bare kommer opp \textcolor{myGrey}{\textit{(viser opp til tavlen)}} og ser på appen, så kan vi forklare litt. Vi skal gå litt kjapt igjennom. Her har vi hjemmemenyen, overblikk som er en slags visualisering \textcolor{myGrey}{\textit{(viser til overblikk bildet)}}. Mine mål som er her \textcolor{myGrey}{\textit{(viser til mine mål)}}, tilpass alternativer som ikke er laget. Her \textcolor{myGrey}{\textit{(peker på walkthrough bildene)}} er en introduksjon til appen, hvor vi har opprettet tre forskjellige \dots det er ting som kan rangeres ifra høyt til lavt, som for eksempel energinivå som kan være lavt eller høyt. Så har vi tidsserier som varer over flere dager, uker eller måneder, som for eksempel et sykehusopphold, en kur du går på, en ny medisin. Så har vi aktiviteter, ting som skjer en dag eller ikke skjer. Som for eksempel tok medisin, var å trente. Så det er det spesielt viktige dager, ekstraordinære dager. Det nederste her er mål, og under det har vi bilder og notater.

\textcolor{myGreen} {p1:}Ja jeg kjenner igjen mye ifra hva vi snakket om sist gang. 
\textcolor{myBlue} {Interviewer:}Ja så dette er på måte et utkast av alt vi gjorde sist gang. Prøvd å sette det i litt struktur. Her har vi \textcolor{myGrey}{\textit{(peker til bildet av hjemskjerm)}} da et forslag til hjemskjerm. 
Det vi skal starte med er å se på hvilke ting dere ønsker å logge her og her \textcolor{myGrey}{\textit{(peker på arket som viser daglig logging)}}. Da kan vi sette oss igjen.

\textcolor{myGrey}{\textit{(Deltakere har fått delt ut et ark hvor de kan fylle ut aktiviteter, tidsserier, mål og andre ting.)}} Ja dere var jo alle med forrige gang.

Dere kan få disse og disse \textcolor{myGrey}{\textit{(Gir deltakere lapper med bilde og tekst av forskjellige målbare ting)}}. Det dere skal gjøre er å dele inn i to bunker med interresant, ikke interresant også etter det igjen skal vi ta og velge de ut seks mest interresante. 
\textcolor{myGrey}{\textit{(Deltakere ser på lappene og deler de ut på bordet)}} Alle bildene er ikke selvforklarende og teksten kan være litt tilfeldig så hvis det er noe dere ikke forstår så er det bare å spørre. 
\textcolor{myGreen} {p1:} Glede\dots Jeg tenker kansje at hvis jeg ikke kan velge imellom den, som er til behandling og hos lege/andre.
\textcolor{myBlue} {Interviewer:} Det er mye det samme
\textcolor{myGrey}{\textit{(De velger kort)}}
\textcolor{myBlue} {Interviewer:}Ja har du valgt ut \textcolor{myGrey}{\textit{(til p2)}}
\textcolor{myYellow} {p2:} Ja det her er de uinterresante.
\textcolor{myBlue} {Interviewer:} Det er ikke utrolig viktig å få det riktig.
\textcolor{myBlue} {Interviewer:}\textcolor{myGrey}{\textit{(Til p2:)}} Har du seks som peker seg ut?
\textcolor{myYellow} {p2:}Ja jeg har det her \dots Kanskje det blir mer enn seks men prøver å ta de beste.
\textcolor{myGreen} {p1:} Hvis man ikke klarer helt \textcolor{myGrey}{\textit{(Spørrende tone)}}
\textcolor{myBlue} {Interviewer:} Å rangere de? Ja bare ta de seks mest interresante og legg resten her.
\textcolor{myGreen} {p1:} Ja det er så mange bra
\textcolor{myYellow} {p2:} Ja klarer ikke helt å velge ut
\textcolor{myBlue} {Interviewer:}Vet du hva vi gjør, vi starter med neste del! Hvis dere bare tar de interresante kortene deres. Ta bort de som ikke er interresante.
\textcolor{myYellow} {p2:} Ja 
\textcolor{myBlue} {Interviewer:} Hvis dere tar de seks mest interresante til å fylle ut denne, så dere kan få en hver \textcolor{myGrey}{\textit{(Deler ut ark til hver deltager)}}.
\textcolor{myBlue} {Interviewer:}På de første tre feltene her kan dere begynne med å fylle ut nummeret deres. På god kveld der
\textcolor{myGreen} {p1:}God kveld 5
\textcolor{myBlue} {Interviewer:}De første tre er jo ting som kan være på en skala, ting som kan rangeres ifra høyt lav til bra. Hvis dere fikk tre slik hver dag, hvilke er viktigst for deg å vite om over tid? Som for eksempel \textcolor{myGrey}{\textit{(viser på arket)}}
\textcolor{myYellow} {p2:}Du tenker følelser og sånn
\textcolor{myGreen} {p1:}Smerte, kan det være
\textcolor{myBlue} {Interviewer:}Ja alt som kan være på en skala ifra høyt til lavt. Du har for eksempel appetit, energinivå, humørnivå.
\textcolor{myYellow} {p2:}Disse tre for eksempel
\textcolor{myBlue} {Interviewer:}Ja hvis du bare skriver de inn her \textcolor{myGrey}{\textit{(viser til feltene på arket)}}
\textcolor{myBlue} {Interviewer:}De tre neste er jo aktiviteter, det er ting man gjør iblant som er greit å holde styr på. Som for eksempel vært til behandling, medisin ller vært på trening. Er det tre ting som peker seg spesielt ut der som dere vil ha med?
\textcolor{myGreen} {p1:} Jeg skriver behandling slash lege jeg hvis det går bra \textcolor{myGrey}{\textit{(blander to kort til et nytt)}}
\textcolor{myBlue} {Interviewer:}Kommer du på noen aktiviteter som\dots
\textcolor{myYellow} {p2:}Behandling\dots skole tenkte jeg å ta andre steder
\textcolor{myBlue} {Interviewer:} Ja det er greit, er det noen andre aktiviteter som du tenkte på. Du trenger ikke å ha alle tre
\textcolor{myYellow} {p2:} Aktivitetsnivå
\textcolor{myBlue} {Interviewer:}Ja den er jo her oppe, men vi kan legge den ned her. Så kan vi se på aktiviteter \textcolor{myGrey}{\textit{(ser på kortene p2 har valgt ut)}}Ja du har jo jobb, tatt medisin. Ja er det noen som peker seg spesielt ut.
\textcolor{myYellow} {p2:}Ja jeg tenker disse to.
\textcolor{myBlue} {Interviewer:}Hvis du skriver de så kan vi gå videre, vi har litt dårlig tid?
\textcolor{myYellow} {p2:}Sånn og så jobb
\textcolor{myBlue} {Interviewer:}På de to neste bildene så står de pågående hendelser som vi visste her istad, som for eksempel \textcolor{myGrey}{\textit{(tar ned bilde av prototypen)}} med ting som skjer. Her har vi skrevet inn proteinkur, paracetamol og nakkesmerter. Den 19 fikk han vondt i nakken og 23 gikk det over igjen. Her kunne det vært begynt på sykehusopphold eller begynt på ny behandling? Er det noen sånne som dere kommer på som ville vært aktuelt for dere? Noe dere ville skrevet inn? Det trenger ikke være noen av disse kortene, det kan være helt egne ting
\textcolor{myGreen} {p1:}Ja hvis jeg skriver ny behandling med notat så kan jeg skrive liksom. Skjønner du hva jeg mener, hvis jeg skriver at jeg er i en utredningsfase for det bla bla bla, for det bla bla bla \textcolor{myGrey}{\textit{(Sier bla bla bla)}}
\textcolor{myBlue} {Interviewer:}Ja så du ville hatt et notat med det
\textcolor{myGreen} {p1:}Ja så jeg ville hatt en oversikt over hvor lang tid jeg har gjodt det.
\textcolor{myBlue} {Interviewer:}Kommer du på noen sånne
\textcolor{myYellow} {p2:}Tatt medisin
\textcolor{myBlue} {Interviewer:}Mener du at starter på en ny medisin
\textcolor{myYellow} {p2:}Ja
\textcolor{myBlue} {Interviewer:}Noen andre ting det kunne vært greit å se igjen i etterkant som varer over flere dager, eller flere uker? 
\textcolor{myGreen} {p1:}Jeg synes at smerte eller energinivå hadde kommet inn i den. 
\textcolor{myBlue} {Interviewer:} Ja så du mener at man er i en positiv periode eller negativ periode. 
\textcolor{myYellow} {p2:} Ja det hører jo sammen med hvordan formen er! Hvis jeg har lite energi så jeg er automatisk ikke så glad. 
\textcolor{myBlue} {Interviewer:}Ja for litt av greie her å bare skrive inn ekstra
\textcolor{myGreen} {p1:} Ja jeg skjønner
\textcolor{myBlue} {Interviewer:} Vi kan jo går videre. Det neste er å legge til bilder som beskriver dagen, her er bilder tatt ifra bildbiblioteket fra den dagen. Tar dere ofte bilder på en dag?
\textcolor{myGreen} {p1:} Nei, men jeg ville gjerne gjordt det? Da hadde jeg blitt mye bedre til det. 
\textcolor{myBlue} {Interviewer:}Ja
\textcolor{myGreen} {p1:} Kan jeg skrive inn et hjerte eller stjerne
\textcolor{myBlue} {Interviewer:} Det er en funksjon du ville ha brukt hvis du hadde?
\textcolor{myGreen} {p1:} Ja veldig?
\textcolor{myYellow} {p2:} Det her er bilder?
\textcolor{myBlue} {Interviewer:} Ja det er bilder ifra telefonen din sånn at du kan legge de til den tidslinjen din!
\textcolor{myYellow} {p2:}Ja jeg er positiv til det!
\textcolor{myBlue} {Interviewer:}Under det er det to punkter. Det første er fullførte du gårsdagens mål, tror jeg det står.
\textcolor{myYellow} {p2:}Målet igår!
\textcolor{myBlue} {Interviewer:}Ja, under det så er det om du har et mål for imorgen. Tankegangen er at hvis du har noe spesielt kan du sette et mål for imorgen! Er det noe dere gjør iblandt.
\textcolor{myGreen} {p1:}Ja det er veldig ofte, du har nye kurer eller ting som skjer og for å holde motet oppe er det veldig ofte at de sier sånn, sett deg et mål for dagen?
\textcolor{myBlue} {Interviewer:}Er det noen spesielle. Hvis dere skriver inn noe dere ville ha skrevet der. Er det noen spesielle mål for dagen dere ville ha kommet på?
\textcolor{myGreen} {p1:} Si at jeg skal trene, eller holde den for imorgen. 
\textcolor{myBlue} {Interviewer:}Siden dere ikke har skrevet inn et mål forigår kan dere hoppe over den
\textcolor{myGreen} {p1:} Ja si at jeg har greid den da så er målet å kanskje\dots Jeg vet ikke, hvis du er skikkelig dårlig kan det jo være så enkelt som å hente posten eller stå opp av senga, spise middag.
\textcolor{myBlue} {Interviewer:}\textcolor{myGrey}{\textit{(Til p2)}}Har du noen
\textcolor{myYellow} {p2:}Ja jeg tenkte skolearbeid
\textcolor{myBlue} {Interviewer:} Da er vi egentlig ferdig med hele den listen. Så hvis du ser tilbake på punktene nedover her. Tenker du deg at ofte eller hver så skal du fylle ut denne som en logg for dagen. Reflekter disse punktene du har skrevet inn på en måte hverdagen din? Er det en god beskrivelse av hva som skjer gjennom dagen din.
\textcolor{myGreen} {p1: }Ja hvis jeg hadde fått inn kanskje jobb at hvis jeg hadde vært på jobb lenge da, så hadde det vært ett mål da.
\textcolor{myBlue} {Interviewer: }Ja her står det ekstraordinære, ved siden av ekstraordinære så er det mål 1 og 2. Hvis vi går tilbake her \textcolor{myGrey}{\textit{(tar ned ark ifra prototypen med mål)}}. Mål her er noe man setter seg over lengre tid! De er ikke ment å være kortsiktige mål som for eksempel å trene imorgen men for eksempel å bli kjent med hvordan man bruker medisin. Vi har lagt til tre forskjellige, det er ting med helse, med aktivitet, personlig utvikling. De kan også være andre ting som skole og lignende. 
\textcolor{myGreen} {p1: }Ja så for eksempel det er det jeg mener, hvis jeg har bestilt meg en reise da. For at jeg har lyst til å reise lenge, men det er ikke alltid det er gjennomførbart for det ikke funker så er det liksom et langsiktig mål, og for å komme dit må jeg kanskje følge medisinene mine, må jeg kanskje trene to ganger i uka.
\textcolor{myBlue} {Interviewer: }Hvis du ser på hele arket er det noe dere kunne tenke dere å fylle hver dag.
\textcolor{myGreen} {p1: }Ja absolutt
\textcolor{myYellow} {p2: }Ja
\textcolor{myGreen} {p1: }Jeg synes det her hadde vært helt genialt å ha i hverdagen.
\textcolor{myBlue} {Interviewer: }Det er kult å høre!
\textcolor{myGreen} {p1: }Jeg tenker at det her kunne vært et verktøy for å nå de målene. 
\textcolor{myGrey}{\textit{(P1 og P2 fyller ut spørreskjema. Med spørsmål om mestringsmål)}}
\textcolor{myBlue} {Interviewer: }Nå har dere fylt ut det spørreskjema med mestringsmål. Hvis dere tenker tilbake på samtalene, har dere hatt samtaler om mestringsmål med helsepersonell?
\textcolor{myGreen} {p1: }Ja
\textcolor{myYellow} {p2: }Ja
\textcolor{myBlue} {Interviewer: }Når dere fikk de skrev dere dette ned på noen måte. Var dere noen måte dere bevarte de? Eller måtte dere bare huske på de?
\textcolor{myYellow} {p2: }Jeg skrev de ned på notatarket på mobilen. Men den brukte jeg aldri.
\textcolor{myBlue} {Interviewer: }Tanken er at jo at hvis du hadde brukt en slik app som dette. Og hatt mestringsmålene beskrevet på en måte som det her. Tror dere det ville hjelpe dere med de?
\textcolor{myGreen} {p1: }Ja
\textcolor{myYellow} {p2: }Absolutt
\textcolor{myGreen} {p1: }Ja jeg tror dette er noe jeg kunne tatt med tilbake til hver legekonsultasjon og sett. Jeg kunne nesten ha sagt håndfast at denne dagen gjorde jeg det og det og det. Det funka ikke, men neste dag så gjorde jeg noe helt annet, da var jeg i god form.
\textcolor{myBlue} {Interviewer: }De målene var det noe dere husket helt klart eller var det mer underbevisst, at dere måtte bli minnet på det av andre?
\textcolor{myGreen} {p1: }Måtte bli minnet på det \textcolor{myGrey}{\textit{(som i det stemmer)}}
\textcolor{myYellow} {p2: }Ja for det var vanskelig.
\textcolor{myBlue} {Interviewer: }Hvis dere hadde blitt minnet på det hver dag tror dere virkelig hadde endret noe eller hadde det vært\dots
\textcolor{myGreen} {p1: }Det hadde hjulpet
\textcolor{myYellow} {p2: }Ja det hadde hjulpet
\textcolor{myGreen} {p1: }Det hadde hjulpet helt sykt
\textcolor{myBlue} {Interviewer: }Nå i etterkant husker dere, hele prosessen som en lang bane. Når dere hadde vært igjennom en mestringsprosess kunne dere tenke tilbake og forklare hva som hadde skjedd innenfor denne prosessen eller var det litt mer uklart.
\textcolor{myGreen} {p1: }Det er jo det rundt deg som må forklare deg og si sånn, husk det for to måneder siden så kunne du ikke det, nå greier du det. Den appen hadde gjordt at jeg kunne sett det mye klarere selv og jeg kunne gått tilbake og sett for eksempel for to uker siden så klarte jeg ikke å løfte en kilo med hånden for da hadde jeg, og nå kan jeg løfte 10. Det var et litt dumt eksempel men du skjønner poenget.
\textcolor{myBlue} {Interviewer: }Tror dere ville ha gjordt dette ofte eller ville dere gått lei av det?
\textcolor{myYellow} {p2: }Jeg ville gjordt det hver uke
\textcolor{myGreen} {p1: }Jeg vilel gjordt det hver dag, ihvertfall nesten daglig. Kanskje ikke like mye i helgene. 
\textcolor{myBlue} {Interviewer: }Det burde vært mulig å fylle ut dager i etterkant?
\textcolor{myGreen} {p1: }Ja, det skulle vært påminnelser hver dag om kvelden. Istedet for å bruke fem minutter på facebook så kunne jeg heller gjordt det her!
\textcolor{myBlue} {Interviewer: }Helt til slutt lurte vi bare om presentasjonen var noe dere likte eller om dere ville hatt det på en annen måte. Om det var forståelig og klart og samtidig om det var noe dere likte.
\textcolor{myYellow} {p2: }Jeg likte den for at den var veldig enkel
\textcolor{myGreen} {p1: }Den ekstraordinært kanskje forklart den på en annen måte eller annet ord. Og for pågående hendelser burde det vært minnet på at jeg ikke trengte å legge til det for hver dag, det burde vært klart
\textcolor{myBlue} {Interviewer: }Da er vi ferdige, tusen takk for hjelpen.s



\end{document}