\documentclass[11pt, norsk, a4paper]{article}
\usepackage{babel,graphicx,textcomp,varioref}
\usepackage[dvipsnames]{xcolor}

%farger jeg skal ha med
\definecolor{myYellow}{RGB}{241, 196, 15}
\definecolor{myBlue}{RGB}{52, 152, 219}
\definecolor{myGreen}{RGB}{46, 204, 113}
\definecolor{myGrey}{RGB}{149, 165, 166}
\definecolor{myR}{rgb}{231, 76, 60}


\usepackage[utf8]{inputenc}
\author{Simon Lysne Hyenes}
\title{Transcription of workshop 2}

\tolerance = 5000
\hbadness = \tolerance
\pretolerance = 2000
\begin{document}
\maketitle{}

\section{Introduction}

\section{Group 2}
This is the transcript from the second workshop. It lasted about 20 minutes and had 2 participants, hereafter called p1, p2 and p3. In the first part the participant are looking at a prototype on a blackboard. The prototype consists of several images that are explained to the participants. 



\textcolor{myBlue} {Interviewer: }Ja, dere var jo alle med forrige gang. Da vi snakket om tidslinjer og tidsforståelse og vi tenkte at vi kunne starte med, hvis dere kommer rundt kan dere se på noen tegninger. Vi har, eller jeg, har laget denne som. Vi kan starte her borte 

\textcolor{myGrey}{\textit{(Viser til prototypen som er på tavle se bilde nr.?)}}. Det her er egentlig bare en instruks for hvordan ting er, her har vi menyen hvor vi har, logg noe, overblikk, tidslinjen, mine mål og tilpasninger. Det er tre forskjellige datatyper som vi skal snakke litt om etterpå. Det er ting som er på en skala som er ifra høy, normal, lav. Det er tidsserier ting som varer over lengre perioder med tid, flere dager. 

\textcolor{myGrey}{\textit{(Ser på hva som står på teksten, deltakere ler)}} Ja det er ting jeg fant på! ville ha med noe. Det siste er aktiviteter ting som man gjør på en dag eller ikke. Til slutt så er det bilder notater, som er tekst og bilder og det er også spesielle for viktige dager og mål. Her har vi en visning som er en forstørret visning av hvordan det ville sett ut hvis du skulle logge en dag, hvor du hadde tre forskjellige ting, både hvordan du føler deg, smertenivå, aktivitetsnivå, her er det medisin, trente og her om det var en ekstraordinær dag, om du jobbet med mål 1 og mål 2 og her er pågående hendelser som kan stoppes og startes. Sist men ikke minst er det notater, legge til bilder ifra bibloteket, om man fullførte målet for idag og man vil sette et mål for imorgen. Så er det en visning her. Her er det en overblikk visning, men det kan vi komme tilbake til! Her er det hvordan det ville vært over flere dager. Her har vi mål, legge til pågående mål og se på fullførte mål som man har gjordt tidligere.

Da skal vi starte workshoppen med å sortere litt. Vi skal se på noen kort som vi har lagt fram her, dere får hver en bunke. Så det jeg ønsker at vi skal gjøre er at dere velger de seks mest interresante av de. Så det er egentlig bare å begynne å bla igjennom. Så dere kan legge de i bunker med interresant, ikke interresant og velge ut hva dere synes er mest interresant. Da tenker vi på ting som ville vært med på de forskjellige visningene.



\textcolor{myGrey}{\textit{(Deltakere sorterer kort, kortene har tekst og ikon)}}


\textcolor{myBlue} {Interviewer: }Bildene er ikke alltid forklarende, så bare spør om det er noen som dere lurer på


\textcolor{myGreen} {p1: }Hva tenker du på med tidstempo


\textcolor{myBlue} {Interviewer: }Om ting går kjapt eller sakte, det er mye det samme som stress. Nå er det kanskje også et kort for stress


\textcolor{myGreen} {p1: }Ja jeg kom akuratt over det 

\textcolor{myGrey}{\textit{(Ler)}} Rett etter den. Jeg legger de to sammen.


\textcolor{myYellow} {p2: }Det var bare interresant ikke interresant.


\textcolor{myBlue} {Interviewer: }Ja og så velge ut de seks mest interresante


\textcolor{myR} {p3: }Ja vi skulle bare velge seks.


\textcolor{myBlue} {Interviewer: }Ja hvis dere har noen spesielle favoritter som peker seg ut som du kunne tenkt deg å logge i en slik applikasjon.


\textcolor{myBlue} {Interviewer: }

\textcolor{myGrey}{\textit{(Til p2)}}Har du funnet noen? Har du flere kan du bare legge de ved siden av. Hvis vi tar ti sekunder til på den! 


\textcolor{myR} {p3: }Ja ta de seks her


\textcolor{myBlue} {Interviewer: }Ja det her er seks mest interresante, da kan vi kaste de andre her 

\textcolor{myGrey}{\textit{(Legger vekk ubrukte kort og sorterer de for alle tre)}} Det her var dine seks og dine


\textcolor{myGreen} {p1: }Ja sånn


\textcolor{myYellow} {p2: }Ja, to ekstra egentlig


\textcolor{myBlue} {Interviewer: }Det vi tenkte på punkt to her er at vi skal se på denne. 

\textcolor{myGrey}{\textit{(Tar frem tre eksempler av arket med logging for hver dag)}}Hvis dere kan fylle inn det dere valgte de, seks mest interresante, Du kan jo ta en penn eller skrive de inn. Når du får inn alle disse seks om det er noe dere kunne tenke dere å gjøre i hverdagen å fylle det ut. På de første tre er det ting som kan være på en skala. For eksempel aktivitetsnivå er en ting som er på en skala, det er jo høyt eller lavt, de samme med energi og humør smerte og søvn.


\textcolor{myGreen} {p1: }Trenger vi å skrive stikkord eller spørsmål til de


\textcolor{myBlue} {Interviewer: }Bare stikkord. Istedet for navnet kan dere bare skrive nummeret dere fikk. Under de tre spørsmålene er det ting som skjer eller ikke skjer på en dag. Som at du foreksempel har vært sosial eller ikke. Du er til behandling for eksempel. Hvis dere ikke har tre går det bra, du har for eksempel bare søvn og stress det går bra. Hvis du har flere går det også bra.


\textcolor{myYellow} {p2: }Ja


\textcolor{myBlue} {Interviewer: }På den neste pågående hendelser er det ting som skjer i hverdagen. Men hvis dere kommer på noe som dere ville ha fylt ut der, som skjer i hverdagen deres som skjer over flere døgn. Er det noe spesielt dere kommer på? Eksemplene her var at personen her har brukt smertestillende i en lang periode og har hatt problemer med blodsukkeret. Sånne ting eller at du har vært lagt inn på sykehus i en lengre periode. Er det noen slike ting som gjentar seg i hverdagen eller som er viktig for dere å logge. 


\textcolor{myGreen} {p1: }Jeg tenker på medisinbruk, da fyller jeg det inn jeg


\textcolor{myBlue} {Interviewer: }Er det noe andre ting dere kommer på, men på tegningen var det kur, nakkesmerte og paracetamol.


\textcolor{myR} {p3: }Kan man skrive to ting


\textcolor{myBlue} {Interviewer: }Ja


\textcolor{myBlue} {Interviewer: }Etter det er det notater ifra dagen om det har skjedd noe interresant. Neste punkt er å plukke ut bilder ifra bildebiblioteket og velge noen av de nyeste. Det er meningen at man kan bla og vist de siste. 


\textcolor{myGreen} {p1: }Ok


\textcolor{myBlue} {Interviewer: }Under det igjen er det mål! Det vi tenkte på er om dere setter daglige mål, som at idag skal jeg gjøre?


\textcolor{myGreen} {p1: }Ja, noen ganger


\textcolor{myYellow} {p2: }Noen ganger. 


\textcolor{myR} {p3: }Ja men det er ikke ofte at jeg tenker sånn at nå skal jeg ha et mål for imorgendagen. Jeg tenker ikke sånn, ikke alltid, men det er noen ganger skal jeg dra på trening eller noe sånt.


\textcolor{myBlue} {Interviewer: }Det siste punktet her er jo det at hvis dere har et mål for imorgen kan dere skrive det ned. Sånn at du har et mål for imorgen. Er det noe typisk dere setter selv? Du nevnte jo trening? Er det noe andre mål dere setter?


\textcolor{myGreen} {p1: }Ja sånn at imorgen har jeg presentasjon på skolen, og da må jeg gjøre det? Det er litt sånn


\textcolor{myYellow} {p2: }Ja sånn eksempel at imorgen skal jeg ha komme igang med en oppgave eller at man er borte på besøk


\textcolor{myBlue} {Interviewer: }Du kan jo skrive det inn


\textcolor{myYellow} {p2: }Ja hele her?


\textcolor{myBlue} {Interviewer: }Du kan bare skrive stikkord. Ja. Du kan også fylle ut på den over det er likegyldig!


\textcolor{myYellow} {p2: }Ja


\textcolor{myBlue} {Interviewer: }Nå har der fylt ut en del på det arket foran dere, hvis dere ble presentert med det som noe som skulle logges hver dag er dette ting som representer dere, føler du?


\textcolor{myR} {p3: }Jeg ville brukt det ganske ofte ja.


\textcolor{myGreen} {p1: }Ja


\textcolor{myYellow} {p2: }Ja trenger det til å reflektere over hvordan man har det?


\textcolor{myR} {p3: }Hvordan man virkelig har det?


\textcolor{myYellow} {p2: }Ja tenkte litt


\textcolor{myGreen} {p1: }Ja tenke på sammenhenger og, hvis du tar stress du ser at du har sovet lite og du ser sammenheng.


\textcolor{myYellow} {p2: }Ja du kan jo, nå er jeg kjempestresset fordi jeg sover dårlig?


\textcolor{myR} {p3: }Ja du kan jo ogås hvis jeg skulle brukt den her så ville jeg blitt flinkere til å sette meg mål egentlig?


\textcolor{myBlue} {Interviewer: }Da kom vi jo på neste del. Under de tre aktivitetene dere skriv inn, så står det ekstraordinær, det er ment hvis dagen inneholder noe spesielt som dere vil at dagen skal utheves. Det er jo visst her 

\textcolor{myGrey}{\textit{(på bildet)}} ved at dagen får et omriss slik at dagen er lettere å plukke ut blandt det andre


\textcolor{myYellow} {p2: }Hvis man for eksempel er ting behandling, ja


\textcolor{myBlue} {Interviewer: }De andre to er mål som er ment å være langsiktige mål. Er det noe dere gjør, setter dere mål?


\textcolor{myYellow} {p2: }Ja


\textcolor{myGreen} {p1: }Ja


\textcolor{myBlue} {Interviewer: }Er det da personlige mål eller hva slags type mål er det normalt å sette for dere? Dere trenger ikke å si hva deres egne mål er men er det skolerelatert, utdanning\dots


\textcolor{myYellow} {p2: }Det kan være hva som helst for min del


\textcolor{myR} {p3: }Det som er veldig fint er å sette mål lengre frem i tid


\textcolor{myYellow} {p2: }Hvis vi tar meg da, har jeg satt meg et mål at når jeg er i thailand skal jeg fullføre dykkelappen og det er et mål som jeg har satt meg


\textcolor{myBlue} {Interviewer: }Hvis vi går tilbake til visningen her har vi tre forskjellige, flere typer mål, helserelaterte, aktiviteter og så har vi personlig mål, som for eksempel å se etter sommerjobber, eller lære seg noe, egenutvikling. Er det noen andre spesielle mål dere setter dere, egenutvikling vil jo være en del av å ta lappen. 


\textcolor{myGreen} {p1: }I framtid vet ikke om det er egenutvikling


\textcolor{myBlue} {Interviewer: }Så lurte vi på, igjennom sykdomsbildet og livet som pasient har dere satt mestringsmål eller satt mål sammen med fastlege eller helsepersonell?


\textcolor{myGreen} {p1: }Ja


\textcolor{myYellow} {p2: }Ja


\textcolor{myR} {p3: }Ja


\textcolor{myBlue} {Interviewer: }Skrev dere de ned, eller var det noe dere fikk med hjem?


\textcolor{myYellow} {p2: }Det ble tatt muntlig, det var mer tenkt på ikke noe spesielt sted det ble skrevet ned! For det kan være fint med noe sånt her da at har man det skrevet og lett tilgjengelig.


\textcolor{myBlue} {Interviewer: }Hvis det da var skrevet ned og dere fikk påminnelelser på hjemmesiden, og når dere fylte ut loggen ville det hjulpet dere eller var dere allerede bevisste på hva målene var. Eller måtte dere minnes på å arbeidet med det?


\textcolor{myR} {p3: }Jeg ville hatt en påminning


\textcolor{myYellow} {p2: }Hvis man hadde hatt en påminning ville man jobbet litt ekstra med det


\textcolor{myGreen} {p1: }Du føler at du må jobbe med det egentlig iallefall sånn som jeg synes


\textcolor{myYellow} {p2: }Sånn som jeg føler at motivasjonen og påminnelser slik at man ikke glemmer målene.


\textcolor{myBlue} {Interviewer: }Til slutt har vi en spørreundersøkelse? 


\textcolor{myGrey}{\textit{(Deltakere fyller ut spørreundersøkelse)}}


\textcolor{myBlue} {Interviewer: }Da er vi straks ferdige og lurte på om dere hadde noen avsluttende kommentarer?


\textcolor{myBlue} {Interviewer: }Syntes dere at presentasjonen er enkel å forstå. 


\textcolor{myR} {p3: }Ja det var veldig rett på sak og tydelig på hva den delen handler om


\textcolor{myBlue} {Interviewer: }Vi har sett for oss at dette er en personlig applikasjon som ikke ville deles med noen er det noe dere ville ha gjordt alikevel? Sånn at dere ville visst det til andre? Er det noe behov for det. 


\textcolor{myR} {p3: }Nei


\textcolor{myGreen} {p1: }Nei, det blir en veldig personlig greie


\textcolor{myYellow} {p2: }Ja, hvis jeg kanskje hadde hatt en første god dag på lenge ville jeg delt noe sånt men det er ikke alt jeg ville delt. Hvis jeg hadde trent for første gang på lenge ville jeg delt det.


\textcolor{myBlue} {Interviewer: }Da sier vise frem, mener du fysisk eller dele på sosiale medier?


\textcolor{myYellow} {p2: }Jeg ville vist frem fysisk ja


\textcolor{myBlue} {Interviewer: }Ville dere da ha en måte å skjule ting i applikasjonen på, slik at andre ikke ser alt?


\textcolor{myGreen} {p1: }Ja


\textcolor{myR} {p3: }Ja


\textcolor{myBlue} {Interviewer: }Er det noe spesielt dere ikke vist frem, er det noe spesielt dere ikke ville hatt med?


\textcolor{myGreen} {p1: }Det kommer veldig ann på humør man er i! Noen ganger er det greit å skjule alt og andre ikke, det er humørmessig og så lenge det går ann å velge selv. 


\textcolor{myBlue} {Interviewer: }Da er vi vist ferdige! Tusen takk for hjelpen det er til stor nytte!.

\end{document}