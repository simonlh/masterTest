\documentclass[11pt,UKenglish, a4paper]{article}
%\documentclass[../../MasterThesis.tex]{subfiles}

\usepackage[utf8]{inputenc}
%--fonts--
\usepackage[T1]{fontenc}
\usepackage[bitstream-charter]{mathdesign}

%--packages--
\usepackage[UKenglish]{babel}
\usepackage{csquotes,textcomp,varioref}

\usepackage{graphicx}
%--color--
\usepackage[dvipsnames]{xcolor}

%Setter inn pdfer
\usepackage[final]{pdfpages}

%-linespace--
\linespread{1.3}

%--hyperlinks-- include 
\usepackage[colorlinks=false, pdfborder={0 0 0}]{hyperref}

%--fullpage--
%\usepackage{fullpage}

%--Uio-Front-Page--
%removed until it works \usepackage{ifikompendiumforside}

%--bibliography- -sortlocale=nb_No,
\usepackage[backend=biber, sortcites, defernumbers, style=numeric-comp, maxnames=2, natbib=true, backref, sorting=none, url=false]{biblatex}
%fjernet ifra style=authoryear-icomp

%farger jeg skal ha med
\definecolor{myL}{RGB}{149, 165, 166}
\definecolor{myR}{RGB}{231, 76, 60}

\definecolor{myYellow}{RGB}{241, 196, 15}
\definecolor{myBlue}{RGB}{52, 152, 219}
\definecolor{myGreen}{RGB}{46, 204, 113}
\definecolor{myGrey}{RGB}{149, 165, 166}
%--Author and Title--
\author{Simon Lysne Hyenes}
\title{Transkribsjon av workshop 1: Gruppe 3}


%--Latex Optimalization--
\tolerance = 5000
\hbadness = \tolerance
\pretolerance = 2000

%---------start------------
\begin{document}
\maketitle{}
\section{Introduksjon}
\textcolor{myR} {p3:}Gruppe 3 hadde tre deltakere. De er markert med tre forskjellige farger og navnet p1, p2 og p3. ``P'' står for participant. 

\section{Gruppe 3}

\textcolor{myBlue} {Intervjuer:} Som tidligere nevnt handler prosjektet mitt om tidslinjer, om det kan være et verktøy for dere å bruke i hverdagen og over lengre perioder. Og er dere kjent med hva tidslinjer er.
alle: Ja

\textcolor{myR} {p3:} Mye om det i historie

\textcolor{myBlue} {Intervjuer:} Ja. Jeg har med noen eksempler bare for å få litt bedre innsikt i hva jeg snakker om. Vi kan starte med den her, dere kjenner det igjen ifra historielinjer Så har vi en litt mer kunstnerisk, den er også historisk.

\textcolor{myR} {p3:} I guriland den var rotetet

\textcolor{myBlue} {Intervjuer:} Ja (ler). Så her er veldig eldre en, nederst her har vi en (type) dere kjenner igjen ifra museer. Så har vi også, dere kjenner kanskje best igjen. Er jo facebook sin tidslinje. Hjemmesiden er jo praktisk talt en tidslinje med handlinger over tid. Vi har noen flere eksempler. (viser treningsappbilder). Det her er min personlige treningstidlinje, så her har du aktivitet over en dag, her har vi oversikt over skritt og bevegelse og andre aktiviteter. Sist men ikke minst har vi søvn. 

\textcolor{myR} {p3:} Det var jo genialt

\textcolor{myBlue} {Intervjuer:} Så har vi noen litt rarere eksempler. Det her er bilder og hendelser på en tidslinje ifra sosiale medier, her har vi en historie beskrevet og tegnet på en tidslinje. Her har er det personlige hendelser og helseting over tid. Så her har vi negativt og positivt og forskjellige ting hun har plottet over flere år. Sist ville jeg bare vise denne, selv om vi snakker om tidslinje trenger det ikke å være en linje, det kan være rundt det kan være andre måter å gjøre det på. \dots Vi tenkte nå i starten at vi skulle tegne litt hvis det går greit med dere. 

\textcolor{myR} {p3:} (sarkatisk) Jeg er veldig flink til å tegne\dots Ikke i det hele tatt, jeg tegner pinnemenn. 

\textcolor{myBlue} {Intervjuer:} Det er absolutt ikke om og men å gjøre dette her pent. Dere kan gjøre det så rotete og vildt som dere vil, dere bare om å gjøre at dere beskriver en tidslinje for dere. Det vi ønsker å se på er enten over en dag, eller en lengre periode som en måned eller et halvt år, eller enda lengre over flere år. Beskrive hendelser og ting som er viktig for dere. Men vi tenkte å relatere det til KULU prosjektet og deres hverdag som pasienter. Hvis det skulle være et verktøy for andre pasienter hva ville vært interresant for de å vite om.

\textcolor{myR} {p3:} Kan jeg låne den\dots

\textcolor{myBlue} {Intervjuer:} Ja hvis dere trenger noe farger eller noe så er det bare å spørre meg, jeg har linjaler, en bøyelig, vi har penner, tusjer, fargeblyanter og alt mulig. Den beste måten å gjøre det på er å sette et punkt et eller annet sted på arket og bare starte. Har dere noen spørsmål?

\textcolor{myR} {p3:} Kan jeg spørre om hva jeg har tenkt?

\textcolor{myBlue} {Intervjuer:} Ja

\textcolor{myR} {p3:} Ja jeg har tenkt på en rett linje hvor du kunne trykke over tid over måned eller sånn, men samtidig gå inn på den dagen, hvordan er det jeg skal forklare det. 

\textcolor{myBlue} {Intervjuer:} Da kan du tegne en linje, og så kan du tegne en linje under som en dag. 

\textcolor{myR} {p3:} Ja for jeg likte det eksempelet!

\textcolor{myBlue} {Intervjuer:} Har dere (henveder til p1 og p2) noen ideer?

\textcolor{myYellow} {p2:} Jeg driver og tenker

\textcolor{myGreen} {p1:} Jeg likte ideen din (p1)

\textcolor{myBlue} {Intervjuer:} Det viktigste er å ha hva dere ønsker å se på over tid. Innenfor prosjektet er det mange ting som er interresant å se på. Jeg kan ta noen eksempler. Som for eksempel legebesøk, bruk av nye medisiner, større hendelser, mindre hendelser, det kan være energinivået ditt, det kan være humør, stress, det kan være starte ny behandling, eller nye venner, kjæreste. Sånne ting. Men det er ikke komplett dere kan finne på hva dere vil og.

\textcolor{myYellow} {p2:} Jeg tenker\dots

\textcolor{myBlue} {Intervjuer:} Ja

\textcolor{myYellow} {p2:} Jeg deler arket i tre.

\textcolor{myBlue} {Intervjuer:} Ja det finnes ikke noe riktig eller galt 

\textcolor{myYellow} {p2:} Ja jeg skjønte at vi skulle tegne en tidslinje men hva skulle den inneholde.

\textcolor{myBlue} {Intervjuer:} Ja hva dere kunne tenke dere å dokumentere for å ha i fremtiden, hva er det interresant for dere å vite som pasienter, som det kunne vært fint å dokumentert. Som pasient på sykehus, hender det iblandt at man får en dagbok hvor man kan føre ting som skjer, legebesøk, behandling. Du kan tegne en tidslinje over en dag på sykehuset, eller en måned, på sykehuset eller i livet ditt.

\textcolor{myYellow} {p2:} Ja det kan jeg ta en måned

\textcolor{myBlue} {Intervjuer:} Ja hvis du tenker det, ja. Dere trenger ikke ha en orginal ide, hvis dere vil bruke et av disse eksemplene så kan dere bruke de, det er helt ok.
note: De tegner

\textcolor{myBlue} {Intervjuer:} Ville også si at det kan være positive ting dere tar med.

\textcolor{myR} {p3:} Er det greit om man skriver litt?

\textcolor{myBlue} {Intervjuer:} Ja, kjør på (om tegningen) det er kjempebra

\textcolor{myBlue} {Intervjuer:} (til p2) Du trenger ikke ta hele månedenen hvis det blir mye.

\textcolor{myYellow} {p2:} Svart tusj

\textcolor{myBlue} {Intervjuer:} ja her, skal vi se.

\textcolor{myYellow} {p2:} Det blir ikke fint da. 

\textcolor{myBlue} {Intervjuer:} Det går kjempebra

\textcolor{myBlue} {Intervjuer:} P3 har du lyst å beskrive hva du har tegner til de andre.

\textcolor{myR} {p3:} Ja jeg kom med litt forskjellige ideer. Den lange tidslinjen er ment som en sånn høydepunktgreie, hvorfor var jeg så høyt oppe den dagen? Her kan man se hvordan det har gått månedene dine eller hva det har vært. Man kan for eksempel trykke inn på den prikken der (punkt på tidslinjen) og se hva skjedde den dagen. Eller hva skjedde der, hva gjorde at jeg var så høyt oppe og hva gjorde at jeg falt ned dit.

\textcolor{myBlue} {Intervjuer:} For å gjøre det tenker du på den dagen (viser til et utsnitt for en dag på tidslinjen)

\textcolor{myR} {p3:} Ja vi kommer til det her som jeg driver og tegner nå, hvor jeg både kan se på dagen min hva jeg har gjordt og hvordan humøret, formen var og i tillegg vil jeg ha en kalender. Så nå driver jeg og tegner alt her. Litt kravstor eller? Kommer på en god, eller er den god!

\textcolor{myBlue} {Intervjuer:} (til p2) Ja. For å spare litt tid kan vi bare ta de første dagene her, hva tenker du?

\textcolor{myYellow} {p2:} Ja jeg er ikke ferdig enda, jeg tenkte det er dagene og så tenkte jeg å dele det i tre og si sånn her og her er det grønt, bruker grøntfarge, og så velger jeg rød farge her, på den siste delen og her er en mellomting. Jeg tenker litt som p3 sier det er noen dager og høydepunkter og andre dager er dårlige dager.

\textcolor{myBlue} {Intervjuer:} Jeg kan jeg tegne det inn

\textcolor{myYellow} {p2:} Ja

\textcolor{myBlue} {Intervjuer:} ja, da setter jeg en grønn prikk her og sier høydepunkter 

\textcolor{myYellow} {p2:} Her kan det være grått (midt på) og her må det være rødt (nederst), og gult her

\textcolor{myBlue} {Intervjuer:} Jeg tegner noen streker, er det det du mente?

\textcolor{myYellow} {p2:} Ja det er det

\textcolor{myBlue} {Intervjuer:} Skal la deg forsette

\textcolor{myYellow} {p2:} Ja for eksempel sånn, så fyller jeg inn tekst her

\textcolor{myBlue} {Intervjuer:} Ja på den dagen. 

\textcolor{myYellow} {p2:} Her er det grønn tekst for en bra dag, og så fyller jeg inn hva som har vært bra. Og så si for dag fire så har det vært dårlig dag så fyller jeg inn rødt og så forsetter jeg sånn. Det er det jeg tenker på. Poenget er at det skal være tre farger og det skal vise forskjell.

\textcolor{myBlue} {Intervjuer:} Ja så du kan være midt mellom her og.

\textcolor{myYellow} {p2:} Ja det er grått.

\textcolor{myBlue} {Intervjuer:} Ja p1

\textcolor{myGreen} {p1:} Ja jeg tenkte litt at det skulle være en egentlig, en eller annen form for kalender. Så kunne man trykke inn på de forskjellige dagene i kalenderen og så kommer det opp forskjellige ting så man kan egentlig skrive på. Hvis det er ting man vil huske på, eller ting man skal gjøre den dagen, så kommer de bildene opp igjennom og så ser man forskjellige bilder aktiviteter og alt nesten som post-it greier eller et annet..

\textcolor{myBlue} {Intervjuer:} Ja. Vi kan gå videre. Et av målene som nevnt er å lage et verktøy som både kan brukes korte perioder og lengre perioder og ment som et verktøy for å kunne dokumentere det dere går igjennom. Og jeg lurte på om dere bruker noen sånn, dokumenterer dere noe i hverdagen idag, for eksempel treningslogg, almanakk, dagbøker, kalendere\dots

\textcolor{myR} {p3:} Skulle ønske at jeg var flinkere til det. Jeg, ungdomsrådet har en kalender og jobben har en, skulle slått de sammen og hatt en.

\textcolor{myBlue} {Intervjuer:} Ja

\textcolor{myYellow} {p2:} Nei jeg er ikke flink med det.

\textcolor{myBlue} {Intervjuer:} Det er ikke noe du gjør heller?

\textcolor{myYellow} {p2:} Nei egentlig ikke

\textcolor{myGreen} {p1:} Jeg liker almanakken jeg. (observatør: Bruker du den) Ja jeg fører, jeg må nesten ha det eller så glemmer jeg så fort ting, det er helt sykt, enten skriver jeg det opp eller så glemmer jeg det. Da får jeg iallefall ikke gjordt ting.

\textcolor{myBlue} {Intervjuer:} Samme her, hvis det ikke hadde vært for det så hadde jeg vært ubrukelig!

\textcolor{myBlue} {Intervjuer:} Vi lurte på har dere som pasienter kunne tenke eller komme på noe hvor dere har hatt behov for å gå tilbake og se på, hva gjorde jeg\dots

\textcolor{myR} {p3:} Veldig

\textcolor{myBlue} {Intervjuer:} i perioden

\textcolor{myR} {p3:} Veldig ofte, jeg har så lyst å kunne gå tilbake og se, det er det jeg prøver å forklare her. Jeg kan forklare hva jeg prøver å tegne da.

\textcolor{myBlue} {Intervjuer:} Ja

\textcolor{myR} {p3:} Ja det jeg mener med kalender er at du kan gå inn på spesielle dager hvor du ser at har skjedd noe spesielt, at det finnes vanlige kalender som iphone men at forsiden er det her, så kan du inn og se, her var jeg så høyt oppe. Hva gjorde jeg der, den dagen? hvorfor falt jeg ned? lære av feilene dine. Samtidig kunne gå inn og se, hva gjorde at jeg holdt meg så stabil der (peker på tidslinjen), da kan jeg se på bilder og at det er blandet med følelser. For følelser og form er to forskjellige ting, ihvertfall for meg, formen kan være ræva og følelsen ganske grei. Men selvfølgelig spiller det sammen. Jeg skjønner at hvis jeg er for glad den ene dagen og bare pusher på så sier det seg selv at jeg (viser til linjen) jeg var kjempeglad her, og så datt jeg ned her, det er ikke rart for jeg gjorde det og det og det og det, det mener jeg at det kan ses sammen. Da kommer vi videre til det her, for da har man tatt bilder og da kan du se på aktiviteten, hvor mye mat har du spist, hva spiste du, så mye gikk jeg, så mye var jeg med venner.

\textcolor{myYellow} {p2:} Jeg syntes det var veldig bra, ihvertfall den her tidslinjen, høydepunkter, dårlige dager, stabile dager, for eksempel her er du på vei opp og ned. 

\textcolor{myR} {p3:} Hva skjedde liksom

\textcolor{myGreen} {p1:} Og så kan du prøve å finne ut hva du gjorde her

\textcolor{myR} {p3:} Ja det er det jeg mener. For da kan man komme seg kjappere oppover for da får man det beskrevet. Det er greit at jeg sier til deg (bruker p2 som eksempel) men det hjelper ikke.

\textcolor{myYellow} {p2:} Jeg syns også de tingene der også bra for der er det flere faktorer som spiller inn på hvordan dagen din blir.

\textcolor{myBlue} {Intervjuer:} Ja jeg lurte på om dere har noen andre faktorer som nevnte som bidrar her.

\textcolor{myYellow} {p2:} Ja søvn, hvor mye du har sovet,  der viktig syntes jeg

\textcolor{myGreen} {p1:} Sosial omgang kanskje

\textcolor{myYellow} {p2:} Ja hvor mye du har vært med venner

\textcolor{myBlue} {Intervjuer:} Vi tenkte på ting som ikke, for mat, søvn og aktivitet er jo noe du ville du kunne fått via en app, men mer på andre som er på sykehuset.

\textcolor{myYellow} {p2:} Jeg ville gjerne ha blodsukker

\textcolor{myR} {p3:} Ja, jeg ville gjerne hatt noe som kunne minnet meg på å ta alle mine tabletter og sånn. En reminder et eller annet sted. Som ikke gi ser.

\textcolor{myBlue} {Intervjuer:} Er det noen andre ting dere vurderte å inkludere.

\textcolor{myYellow} {p2:} Ja jeg tenkte, du kunne skrive her om du har tatt medisiner

\textcolor{myR} {p3:} Ja egne påminnelser til tabletter og avtaler! I kalenderen. Jeg ville ha noe som jeg kunne sett på. Som å se tilbake på det her, så kunne jeg sett at i sommer var jeg ute med venniner mine og jeg var glad og formen var bra og jeg veit at jeg kan komme meg dit igjen det har veldig mye å si da. Da ser jeg hva som funker.

\textcolor{myBlue} {Intervjuer:} Ja sammen istad hadde dere en fri-fantasi fase. Jeg lurte på hvis dere tenker på det samme her men bare innenfor helt fritt fantasi er det noen spesielle ekstra ting dere vil legge til da. Er det noe dere kunne gjort?

\textcolor{myYellow} {p2:} Video, bilder kanskje, jeg vet ikke om det er nødvendig for det er jo bilder som

\textcolor{myR} {p3:} Ja har det med her

\textcolor{myYellow} {p2:} Det er jo bilder som viser fortiden din

\textcolor{myR} {p3:} Hvis jeg er innlagt angrer jeg på at det er ting jeg ikke får med meg, for eksempel hver gang jeg er innlagt kommer legene inn og så er jeg på for mye medisiner så jeg husker aldri noen ting av hva de snakker om på legebesøkene hvis jeg er innlagt og de går visitt. Da er det helt kake og da husker jeg ingenting. Da hatt en greie hvor jeg bare trykket på telefonen og trykket på opptak, selv om det finnes på telefonen men da måtte jeg ha bladd igjennom. Istedet husket, 23. november da var jeg hos legen min og da høre hva han sa. Hva sa han? Så kunne jeg tatt opp hele samtalen?

observatør: Det er ikke lov

\textcolor{myBlue} {Intervjuer:} Ja vi er på fri fantasi.

\textcolor{myR} {p3:} Da skriver jeg et hjerte bak jeg!

\textcolor{myBlue} {Intervjuer:} Et eksempel var på disse punktene, tenkte du notater eller (til p2)?

\textcolor{myYellow} {p2:} Ja egentlig var det kun notater, sånn dagbok da, det var ikke mer enn det!

\textcolor{myBlue} {Intervjuer:} Ja så det post-it, følger bare om det er en bra dag eller en dårlig dag?

\textcolor{myYellow} {p2:} Ja bare dårlig bra, jeg tenkte ikke lengre enn strukturen altså formen på. Så det var ikke mer enn det.

\textcolor{myBlue} {Intervjuer:} Før vi avslutter var det noe mer

\textcolor{myGreen} {p1:} Ja tenkte på noe, det var litt sånn smak og behag for forskjellige folk. Men hvis man har andre venner som også bruker appen så kunne man legge dem til og se, eller dele eller hva enn man vil.

\textcolor{myR} {p3:} Så kunne jeg og p1 gitt hverandre hints og ideer.

\textcolor{myGreen} {p1:} Det er jo selvsagt ikke noe så må du må gjøre men

\textcolor{myYellow} {p2:} Det er jo en ting at det er en privat sak.

\textcolor{myR} {p3:} Ja at du lukker den da, men at hvis jeg vil dele en dag og la andre seg hva jeg gjorde

\textcolor{myYellow} {p2:} Ja hvis de godtar det

\textcolor{myBlue} {Intervjuer:} Vi må desverre avslutte men tusen takk for hjelpen, det har vært kjempeinterresant og dere har vært kjempeflinke. Tusen takk.

\end{document}