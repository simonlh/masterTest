\documentclass[11pt,UKenglish, a4paper]{article}
\usepackage[utf8]{inputenc}
%--fonts--
\usepackage[T1]{fontenc}
\usepackage[bitstream-charter]{mathdesign}

%--packages--
\usepackage[UKenglish]{babel}
\usepackage{csquotes,textcomp,varioref}

\usepackage{graphicx}
%--color--
\usepackage[dvipsnames]{xcolor}

%Setter inn pdfer
\usepackage[final]{pdfpages}

%-linespace--
\linespread{1.3}

%--hyperlinks-- include 
\usepackage[colorlinks=false, pdfborder={0 0 0}]{hyperref}

%--fullpage--
%\usepackage{fullpage}

%--Uio-Front-Page--
%removed until it works \usepackage{ifikompendiumforside}

%--bibliography- -sortlocale=nb_No,
\usepackage[backend=biber, sortcites, defernumbers, style=numeric-comp, maxnames=2, natbib=true, backref, sorting=none, url=false]{biblatex}
%fjernet ifra style=authoryear-icomp

%farger jeg skal ha med
\definecolor{myY}{RGB}{241, 196, 15}
\definecolor{myB}{RGB}{52, 152, 219}
\definecolor{myG}{RGB}{46, 204, 113}
\definecolor{myLy}{RGB}{149, 165, 166}
\definecolor{myR}{rgb}{231, 76, 60}

%--Author and Title--
\author{Simon Lysne Hyenes}
\title{Transkribsjon av workshop 1: Gruppe 2}


%--Latex Optimalization--
\tolerance = 5000
\hbadness = \tolerance
\pretolerance = 2000

%---------start------------
\begin{document}
\maketitle{}
\section{Introduksjon}
Gruppe 2 hadde tre deltakere. De er markert med tre forskjellige farger og navnet p1, p2 og p3. ``P'' står for participant. 
\section{Gruppe 2}
i: Jeg heter Simon forresten bare sånn at det er sagt. Oppgaven her handler om tidslinjer og om det går ann å bruke tidslinjer som et verktøy i hverdagens deres til å få oversikt over ting som skjer over korte perioder og også over lengre perioder, måneder eller år. Siden det er KULU prosjektet så tenker vi relatert til helse og dere som pasienter, men også litt friere og. Så først skal vi starte med \dots om dere vet hva tidslinjer er? Hva jeg prater om. 
p1: Ja om det sånn tid i perioder som vi har hatt på skolen. (imiterende) \textit{Her startet hele prossesen før annen verdenskrig}. Så tegner man en sånn prosess hvor ting skjer hvert år
i:Ja
p1: Så skal det være eksempel over dager eller
i: Absolutt
p2: Jeg tenkte mer på
P3: Jeg tenker mer på streker
i: Jeg har med noen eksempler. Vet du. Vi har jo sånn klassiske (viser tidslinje med historiske gjenstander) hvor det er massevis av ting over lengre perioder.
p3: Bra tidslinje ville jeg si.
i: Og så har vi en veldig gammel hvor det bare var tekst og..
p2: Fy søren (imponert)
p3: Sånn vil vi ikke ha (spøkende)
alle: (latter)
i: Så er det den klassiske historien da, 
p2: Musikkhistorien
p1: Nei (?)
i: Her er et bilde fra en nettside hvor tidslinjen beveger seg innover. Så du scroller opp.
p1: Kult
p2: Fancy
p3: Dritkult
i: Så er det da Facebook, som jeg regner med at dere er kjent med. Hjemmesiden er jo også en tidslinje. Bare at den er horisontal ikke. Så\dots vi har noe flere eksempler her vi skal gå kjapt gjennom de. (Viser bilder ifra Jawbone Up appen med aktivitet, søvn og trening) Det er her er min personlige\dots eller en treningsapp jeg bruker som viser oversikt over aktivitet i løpet av en dag. 
p3: Oi. Hva heter den appen der.
i: Det er en jawbone up.
p1: Det minner meg litt om\dots
p3: (kjapt)nike
p1: fuelband ja
i: Ja det er en sånn (intervjuer viser at han har på jawbone up24 båndet). Det er bare for å vise
p3: sånn (\textcolor{myLy}{umulig å høre}). 
alle: (latter)
p3: VI har hatt det alle vi gutta i klassen.
i: Ja
p3: (spørrende) Hvem sover mest?
p1: Det kunne ikke\dots
p3: Ja, (fleipende) det er ikke meg
p1: Nei!
i: Skal vi ta de fire siste her. Det her er en sosiale medier oversikt (nettsted) med bilder, hendelser og steder over tid.
p3: Kult
p2: Kult
i: Så her er det bare for å vise (viser sirkeltidslinje) at det ikke trenger å være en linje det kan være en sirkel eller andre ting. \dots. Så det to siste her er personlige, det er ikke mine, men den ene er en historie ifra en person som forteller om hvordan hun ble en lærer (viser skriblet tidslinje) over tid. Den andre er en oversikt over følelser og hendelser i livet til (viser ``Amelia'' sin tidslinje). Her er positive ting (peker øverst) og negative ting (peker nederst) fordelt over.
p3: Den var stilig.
p1: Ja, litt rotete men stilig.
i: Ja så tankegangen var at vi tenkte å la dere starte med å få tegne. For å beskrive litt av deres, det er viktig her at dere\dots det er ikke noen riktig måte å gjøre det her på. Det er ikke noen feil måte å gjøre det på heller og det trenger ikke være pent. Det trenger ikke tegne rette streker, eller bruke farger, dere kan rote det til. Bare at dere beskriver for dere. Så er det da, det kan jo være, det kan jo gjøre hva dere vil, men det er lettere hvis dere tenker på at det skal være et verktøy for andre pasienter eller for pasienter som starter å bruke verktøyet skulle ha en lengre tidsoversikt over hele pasientforløpet. 
p1: Hmmm
i: Men skal vi prøve å starte. Det er jo massevis av tegneutstyr her. Det er bare å spørre så kan jeg hjelpe dere. 
p3: Ja
p1: \dots
p3: (fleipende) Jeg er jo knakende god til å tegne
i: Vi har linjaler, som er bøyelige\dots
p2: Den var dritkul
p3: Kult
i: Vi har svære tusjer, vi har mindre, vanlige penner og fargeblyanter. Så hvis dere ikke vet hvordan dere skal starte er det bare å sette et punkt på arket et eller annet sted. Og så er det egentlig bare å. 
merknad:(gruppen ser undrende ut)
i: Ingen av dere er klare eller ingen av dere har noen\dots
p3: Nei, jo, neiiii 
p1: Nei
p2: Ikke noe akuratt\dots
p3: Ikke der (klapper)
i: Da kan jo vi starte med noen eksempler, det her er ikke ment å være riktig men bare noen ting vi har tenkt på. Som for eksempel legebesøk, over tid. Man kan ha oversikt over sykdom, som dagsform eller innleggelse. Oversikt over energi, humør, stress i hverdagen\dots medisinbruk, nye leger, sykehusopphold. Det her var bare noe, det trenger ikke være det
p3: Ikke bare sjukehus det er snakk om
i: Nei det trenger ikke bare være det, det kan også være om hvordan dere personlig har det eller større hendelser i livet deres eller andre ting. 
p3: Ja
i: Så det første å starte med, er jo om dere ønsker å ta et år, ønsker dere å ta flere år. Ønsker dere å ta en dag.
p3: Ja, jeg hadde tenkt en måned jeg ass\dots
i: Ja
p3: Bare sånn
i: Absolutt
p2: Jeg tenkte et halvt år
p3: Jeg har brukt det mer som en møteplanlegger
p2: (ler)
p3: (spørrende)Hva gjorde jeg i år
i: Vil dere ha noen penner eller 
p3: (tok en penn) Ja rosa er fint vet du.
p2: Blir mye rosa idag
p3: (spøkende) Er glad i rosa. (enda mer spøkende) Var yndlingsfargen min da jeg var liten
p2: Seriøst (deadpan)
p3: Ja. Ikke min stolteste innrømmelse\dots men
p2: Det er jo lov med litt
p1: Jeg tror ikke jeg kommer på med noen deler. Det er liksom på en måte hva man gjør i løpet av dag da eller
i: Ja\dots hvis du skulle tenke deg en dag hva er det som er interresant å vite mer om i fremtiden eller gå tilbake og se på i senere tid
p1: Ja ok da ble det jo litt mer vanskelig. Hva var det du tok\dots en måned
p3: Ja
p2: Jeg tok seks måneder
i: Andre har tatt flere år også 
p1: Jeg tar et år
p3: Jeg har lyst å bare gjøre det sånn
i: Ja det er jo ikke noe som er riktig, det er bare å kjøre på 
merknad: Dårlig tid
p3: Jeg aner jo ikke hva jeg driver med
i: Hvis det gjør noen feil så bare stryk over, snu arket eller\dots
p1: Jeg skjønner ikke helt hva jeg skal tegne for et år
p2: Velg en ting. Jeg tenker en smerteskala
p1: Nesten som en sånn der\dots
i: Bruker du en dagbok eller noe\dots eller fører en almanakk
p1: jeg har jo en kalenderbok ting ja. Jeg skriver jo bare sånn som er viktig å huske. Men for et år ser er jo det? Om det er en prøve\dots onsdag 3. desember (tilfeldig valg) er det da viktig for senere\dots
i: Ja du kan jo tenke deg nå. Er det noen tidspunkter i livet ditt som det hadde vært fint å dokumentert nå i etterkant så du kunne gå tilbake og tenkt mer på eller sett på. Som for eksempel i denne måneden her skjedde disse tingene jeg gjorde disse hendelsene.
p1: (sukker!)
i: Det er vanskelig altså\dots det er ikke\dots
p1: Så hvis jeg tar (år) kan jeg bare tegne opp et helt år
merknad: (jeg var usikker på om p1 forsto meg og om jeg hadde rotet det til for p1. Ønsket derfor å dedikere mer tid til de andre)
i: Det trenger jo ikke være lineært det kan jo være at juni tar halve arket og resten er helt fritt
merknad:(latter. p3 gjorde noe morsomt)
p3: Ja, ja, ja jeg skal ikke bli ingeniør kjenner jeg
i: Du kan jo beskrive hva du starter med
p3: Ja jeg har jo satt opp måneden som en sirkel da
i: Ja
p3: Så har jeg fargekode på og du styrer i appen hvor lang periode den skal være også velger den hvor stor del av sirkelen den skal ta. Så hvis det her er en tredagers periode (et utvalg av sirkelen) som har vært bra så fyller den så og så mye av sirkelen og så kan da man legge inn om det har vært en bra periode eller om du har gjordt en ting eller om du har trent mye i en periode. 
i: Interresant
p3: (spørrende tone) Jeg vet jo ikke hva jeg driver med
i: Du kan jo også tegne hva som skjer når du tegner når du trykker inn på et punkt som her
p3: (undrende) Ja hva skjer da
i: Eller hvis du ikke ønsker å tegne kan du bare skrive det innenfor det punktet
p3: Lurt
merknad: (De tegner)
p3: Hvis jeg skal lage en underside så gjør jeg bare sånn da. 
i: Ja
merknad: (De tegner)
i: (Snakker til p2) P2 du tegner en en linje med?
p2: Jeg tegner en sånn (peker på linjen) en sånn seks måneder med smerte, med smertetopper etter hvor mye smerte man har
i: Ja
merknad: (De tegner)
i: (Til p3) Skal du ha noe?
p3: Ja en blyant hadde vært fint
i: Hva tenker du p1?
p1: Jeg tenker at det her (ett punkt) er en måned i ett år så kan man på en måte\dots jeg vet ikke\dots trykke på en av de også kommer det opp sånn for eksempel, positive ting som skjedde den måneden
p2: Jeg går for veldig enkelt
i: Ja (til p1) hvis tar en måned og tegner en strek opp ifra den (ett punkt) hvis du trykker inn er det positive ting er det noen spesielle ting du tenker på?
p1: Nei for eksempel\dots hvis det var en veldig spesiell hendelse som var veldig veldig ålreit (?). Jeg kan jo ta Juli!
i: Ja det er absolutt en fin måned.
p1: Jeg tok dykkerlappen og det var ganske kult
p3: Jeg er skikkelig misunnelig på at du får lov til å dykke. Jeg blir nekta å dykke. Faren min var profosjonell dykker så jeg har lyst å bli dykker selv og de sa at du svømmer som en bryne(?) du kan jo ikke ta dykkerlappen du
p2: Vet du hva jeg så det var en undervannsrullestol! 
p3: Med oksygentank?
p2: Ja den var bakpå\dots og små propeller underpå. Den var dritkul!
p3: Ja da hadde jeg følt at jeg var sånn James Bond!
p2: Det så sånn ut. Da var noe av det kuleste jeg har sett
merknad: (De tegner)
i: Ja så p2 du har beskrevet nivåer.
p2: Ja det er måneder men jeg hadde tenkt jeg skulle beskrive nivåer også
i: Ja
p2: Vet ikke helt hvordan enda!
i: Nei (forstående)
merknad: (De tegner)
i: Ja (til p2) smerten er det noe du tenker at du ville ha skrevet inn selv?
p2: Ja, det er noe du skriver også former den dette her mønsteret utifra hva du skriver inn! Du tar skalaen ifra 1 til til 10 også tar den og lager dette mønsteret ifra det.
i: Er det interresant å kombinere det med noe annet. For hvis du har oversikten her\dots
p2: Ja jeg tenkte aktivitetskala også, for å hvor mye vondt man har ifra hvor mye man gjør
p3: Ja skal vi se det ble jeg skikkelig gæren av alt den rosa fargen
merknad: (latter)
i: (fleipende) Rosa var et dårlig valg
p3: Ja
p1: Du angrer nå (spøkende)!
p3: Ja, sånn skikkelig
i: Da skal vi ikke ha den helt rosa
p3: Ja ta den i nøytral blå eller et eller annet!
merknad: Gir nøytralblå pen til p3
p3: Ja der har du den!
i: Ja vi kan jo ta et minutt til på det her, så kan vi gå videre. Det er ikke veldig viktig å få det riktig, det er bare for å få dere til å tenke på det og tenke i rammene av hva prosjektet handler om
merknad: De tegner
i: Jeg tenkte\dots sånn til å starte med er tidslinjen ment å være et verktøy som kan vise og dokumentere over lengre perioder og dere hadde jo en fri-fantasi fase (før workshop) sammen med Johan (annen student) istad. Innenfor de samme rammene hvor det ikke er noen begrensinger teknologisk, etisk eller noe som helst. Kunne dere tenke dere noen spesielle andre ting å inkludere for et sånt verktøy? 
p2: Innenfor en tidslinje?
p3: Innefor helse eller generelt?
merknad: Undrende blikk!
i: Vi kan faktisk bare komme tilbake til det. Vi kan begynne på en annen. Jeg så nå et jeg hadde hoppet over et spørsmål skjønner du! Beklager det. 
p3: Det går fint!
i: Vi lurte først på om dere dokumenter hverdagen på noen måte? Om dere bruker for eksempel almanakk, blogg, dagbøker, treningslogg eller noe sånt. Er det noe dere gjør på slutten av kvelden for å få noen oversikt senere.
p3: Jeg er plikta til å skrive treningsdagbok. Det er dritkjedelig men jeg må skrive det, jeg har det.
p1: Jeg har jo kalenderbok, det eneste jeg gjør er å stryke alt jeg har fått gjordt og så føre videre det jeg ikke har fått gjort til en annen dag
i: Kan dere huske noen tilfeller hvor dere kunne tenke dere å ha et slikt verktøy for å gå tilbake og se på noen perioder.
p2: Hvis du skal til legen og så spør dem hvordan har
tid: 0542

\end{document}


