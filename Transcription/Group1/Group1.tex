%\documentclass[11pt, norsk, a4paper]{article}
\documentclass[../../MasterThesis.tex]{subfiles}

%farger jeg skal ha med
\definecolor{myYellow}{RGB}{241, 196, 15}
\definecolor{myBlue}{RGB}{52, 152, 219}
\definecolor{myGreen}{RGB}{46, 204, 113}
\definecolor{myGrey}{RGB}{149, 165, 166}

\author{Simon Lysne Hyenes}
\title{Transcription of workshop 1}

\tolerance = 5000
\hbadness = \tolerance
\pretolerance = 2000
\begin{document}
\maketitle{}

\section{Introduction}

\section{Group 1}
This is the transcript from the first workshop. It lasted about 20 minutes and had 2 participants, hereafter called p1 and p2.

\textcolor{myBlue} {Interviewer:} Prosjektet mitt handler da om tidslinjer om det er mulig å bruke det som et verktøy for å gi bedre innsikt eller oversikt over både hverdagen og over lengre perioder. Dette prosjektet er helt i oppstarten. Er dere kjent med tidslinjer vet dere hva jeg mener da?

\textcolor{myGreen} {p1:} Altså sånn som den på facebook da?

\textcolor{myBlue} {Interviewer:}Ja. 

\textcolor{myGreen} {p1:}. Hvor man har en sånn timeline over alt man har postet eller gjordt!

\textcolor{myYellow} {p2:} Hmm

\textcolor{myBlue} {Interviewer:}. Ja absolutt. Jeg har tatt med noen eksempler her. Det er mye men dere kjenner kanskje igjen ifra skolen. Her er facebook (tar fram facebook bildet og legger det bort igjen), ifra skolen og museum er dere kanskje kjent med tidslinjer med objekter.

\textcolor{myGreen} {p1:} Ja

\textcolor{myYellow} {p2:} Mm (som ja)

\textcolor{myBlue} {Interviewer:}. Facebook er et godt eksempel på en sosial tidslinje som beskriver \dots over tid da. 

\textcolor{myYellow} {p2:} Mm (som ja)

\textcolor{myBlue} {Interviewer:}. Her har du et mer kunstnerlig og en historisk tidslinje hvor vi ser ting beveger seg litt fritt og det er litt mer flyt. Jeg går litt kjapt hvis det går greit. Her er en ifra internett hvor man kan bevege tiden sånn at man ser en liten brøkdel om gangen. 

\textcolor{myGreen} {p1:}. Ja.

\textcolor{myBlue} {Interviewer:}. Her har vi noen litt anderledes eksempler. Her nede. Her er en person som har satt opp en tidslinje med hendelser med helsen sin over en lengre periode. (viser frem amilie sin). Her er negative ting (peker mot nederst) og positive ting (peker øverst). Det er satt opp over tid.

\textcolor{myYellow} {p2:} Ja så smart.

\textcolor{myBlue} {Interviewer:} Som prøver å ja

\textcolor{myYellow} {p2:}Det var kjempesmart

\textcolor{myBlue} {Interviewer:}Det trenger ikke være en rett strek, her er en bare noen som har markert årene ved hver sving (viser bilde). Det kan også være satt sammen med bilder og hendelser, det trenger ikke bare være tekst eller historie. 

\textcolor{myYellow} {p2:} Man kan ha med bilder og sånn. 

\textcolor{myBlue} {Interviewer:}Ja absolutt. Så det siste her er bare for å vise her at det trenger ikke være linje i det hele tatt det kan være en rund sirkel. Så friheten er stor.

\textcolor{myYellow} {p2:} Er det litt sånn. Jeg ser står bare april. 

\textcolor{myBlue} {Interviewer:}Ja.

\textcolor{myYellow} {p2:} Det er det bare \dots

\textcolor{myBlue} {Interviewer:}Den er valg der (markerer)

\textcolor{myYellow} {p2:} Ja der ja, jeg trodde det var bare sånn april og så. 

\textcolor{myBlue} {Interviewer:}Så er det også treningsverktøy der bruker man ofte tidslinjer. Det her er min personlige, med bilder ifra telefonen min. Som viser \dots Ja dokumentert trening, bevegelse og søvn.

\textcolor{myGreen} {p1:} Jeg har jo. Jeg så jo et tilfelle av tidslinje idag. På en sånn here læringsportal. Codeacademy bruker det og longterna (?) hvor man kan se hvilke dager man har vært og jobbet. Etter å ha gjordt oppgaver da. Så har man tidslinjer hvor det står sånn herre \dots på mandag gjorde du så mange oppgaver og på tirsdag gjorde du ingen oppgaver så får du et poeng for hver dag du er aktiv og den nettsiden og det går sikkert ann å bruke på andre områder og måter da.

\textcolor{myBlue} {Interviewer:}Ja 

\textcolor{myGreen} {p1:} Du har jo, nettsider som skal belønne deg for å være aktiv. Så kan du på en måte se tidslinjen over. Hvis du har vært aktiv i veldig mange dager på rad. Så er det jo litt kjipt å gå glipp av en dag. Der da.

\textcolor{myBlue} {Interviewer:}Ja for det er et hull i linja.

\textcolor{myGreen} {p1:} Ja det er veldig irriterende for å si det sånn. For sånne som meg som er opptatt av sånne ting på sånne tjenester jeg har brukt hver eneste dag og det er en sånn liten linje oppi hjørnet, så ser jeg at hvis jeg ikke logger inn i morgen så får jeg en belønning. Hvis jeg ikke gjør det så får, så jeg starte på nytt

\textcolor{myBlue} {Interviewer:} \textbf{Ja sorry (så på notater). Jeg er helt enig med deg. Ja poenget her er at vi ser jo på tidslinjen som vi har tenkt å vinkle mer mot dere. Som personer og mer mot deres helse da. Så tidslinjen skal hjelpe dere med helserelarterte problemer eller andre på ahus ungdom og yngre. Tanken er at du kan bruke det ifra du starter opp som pasient og kanskje ifra du er tolv til du er ferdig eller i transisjon. Så jeg tenkte vi bare kunne starte vi egentlig. Det jeg har lyst å gjøre først her er bare å tegne. Da tenkte jeg vi kunne prøve å tegne en tidslinje hvor dere beskriver selv noenting. Det er fritt å gjøre, dere kan enten ta en dag men det er også fint hvis dere tar lengre perioder som en måned, eller lengre en ett halvt år. Men bare beskrive noen av de tingene som er viktig for deg. Noe du kunne tenke deg å ha oversikt over, i helse hverdagen din. For eksempel.}
ß›
\textcolor{myYellow} {p2:} \textbf{Kunne jeg for eksempel tatt min sykehusperiode da.} 

\textcolor{myBlue} {Interviewer:}Ja.

\textcolor{myYellow} {p2:} Kunne jeg ha tatt det som et eksempel her.

\textcolor{myBlue} {Interviewer:} \textbf{Ja. absolutt. Det er veldig fint. Det vi ser er dine egne ting beskrevet på egne måter.} 

\textcolor{myYellow} {p2:} Ja, men skal vi liksom, på en måte designe vår egen hver eller skal vi liksom ta en.

\textcolor{myBlue} {Interviewer:}Ja det er fint du spør. Det er helt opp til dere hva dere gjør. Det beste er jo at dere finner på noe. Men hvis dere ikke, sliter eller trenger hjelp så er jeg her. Jeg kan lage selv eller hjelpe. Det beste egentlig er bare å starte et eller annet sted på arket. Bare sette et punkt.

\textcolor{myGreen} {p1:} Ja men må jeg ha alle årstallene.

\textcolor{myBlue} {Interviewer:}Ja, på en tidslinje kan man ofte sette det opp tider og gjøre 

\textcolor{myGreen} {p1:} Ja men intervallene er litt opp til meg

\textcolor{myBlue} {Interviewer:}Det er helt opp til dere. Vi har tusjer, litt større tusjer. Vi har penner, blyanter.

\textcolor{myYellow} {p2:} Du har ikke noen blyant (henter blyant)

\textcolor{myBlue} {Interviewer:}Ja. Det trenger ikke være sånn at det er riktig intervaller. Så har jeg også linjaler og den (bøyelige linjal) kan du forme som du vil.

\textcolor{myGreen} {p1:} Ja kult.

Lengre pause. 

\textcolor{myBlue} {Interviewer:}Jeg ville bare si at det ikke er viktig at det er pent, at det er fargerikt eller noe sånt. Det er bare å gi dere en mulighet til å uttrykke for dere selv da. \dots Det er fint om du beskriver hva du tenker.

\textcolor{myYellow} {p2:} \textbf{Ja jeg tenkte å lage en vei, ifra start til. Da har jeg startpunkt der hvor jeg først ble syk. Så her går det på en måte en vei. Hvor det er sånn oppturer og nedturer.}

\textcolor{myBlue} {Interviewer:}Ja absolutt. Det er kjempebra

\textcolor{myYellow} {p2:} Ja det er det jeg tenkte på.

\textcolor{myGreen} {p1:} \textbf{Jeg tenkte litt av dette samme. Jeg er jo ifra buk så jeg har jo ikke hatt noe sånn fysiske sykdommer. Så jeg tenkte litt mer å kartlegge hvordan jeg har utviklet meg de siste trene årene.}

\textcolor{myBlue} {Interviewer:}Ja absolutt, det er kjempebra (lys tone. usikker?)

\textcolor{myGreen} {p1:} \textbf{Det kan være interresant for meg selv å gjøre det her. Hvor mye tid har vi. }

\textcolor{myBlue} {Interviewer:}Ja vi tenkte å bruke opptil 5 minutter til på tegningen. For min del kan dere tegne så røft dere vil. 


\textcolor{myGrey}{Note:} De tegner

\textcolor{myBlue} {Interviewer:}Hvis dere trenger noen farger så bare si ifra.

\textcolor{myGreen} {p1:} Kan jeg få en rød, eller mørkebrun.

\textcolor{myGrey}{Note:} Gir over en blyant

\textcolor{myBlue} {Interviewer:}\textbf{Hvis det er vanskelig å beskrive hendelen i detalj så kan dere bare markere den av.} 

\textcolor{myYellow} {p2:} \textbf{Det er ikke det at jeg skriver. Jeg har tegnet lei seg fjes og smilefjes. For i starten og sånn var det en tung periode. Så bare bygger det seg oppover \dots både oppturer og nedturer. Jeg har ikke tenkt over at oppoverbakker (tegningen) er..}

\textcolor{myBlue} {Interviewer:}Ja det er bare en sti

\textcolor{myYellow} {p2:} Ja det er bare en vei. Så ja.

\textcolor{myBlue} {Interviewer:}Så skjer det ting langs veien.

\textcolor{myGrey}{Note:} De tegner

\textcolor{myYellow} {p2:} Halveis ferdig

\textcolor{myBlue} {Interviewer:}Ja skal bare si at vi tar bare to minutter til. For det viktigste er ikke at dere fullfører det her men at..

\textcolor{myYellow} {p2:} Ja at det er en ide.

\textcolor{myBlue} {Interviewer:}(snakker til 22) Litt sånn som meg, liker å bytte ut farger og \dots

\textcolor{myGrey}{Note:} De tegner

\textcolor{myBlue} {Interviewer:}\textbf{Ja da tenker jeg at hvis det er greit så går vi videre. Skal vi se. \dots Nydelig det er kjempespennede. Lurte på om du kunne beskrive litt mer hva du har tegnet.}

\textcolor{myYellow} {p2:} \textbf{Her starter vi i (dato utelatt) da fikk jeg diagnosen. Så i (utelatt) startet jeg med behandling og så oppover første operasjon (dato utelatt). Da er vi forsatt inne i (år utelatt) og så går det over i (neste år). De forskjellige årstallene hva jeg gjorde da.}

\textcolor{myBlue} {Interviewer:}Du viser noen smilefjes her og noen sure fjes. Beskriver det\dots

\textcolor{myYellow} {p2:} \textbf{Det beskriver istedet for at jeg skriver det så er det kanskje at jeg hadde en dårlig dag. Hadde en dårlig tid. Dager jeg hadde det kjempebra, så positivt på ting, det varierte veldig da. Og selv om jeg nærmet meg mål kunne jeg ha det tungt uansett. Men jeg ville jo på måte sette et smilefjes (peker mot slutten av tidslinjen). (latter)}

\textcolor{myBlue} {Interviewer:}Da lurte jeg på i dine egne ord (spør 22)

\textcolor{myGreen} {p1:} Det er litt av det samme hvordan det startet og hvordan det utviklet seg over tid. Jeg kunne vel egentlig skrevet veldig mye mer her men det her er på måte, sinns, humør.

\textcolor{myBlue} {Interviewer:}Ja så det er humør over forskjellige tidsperioder. Mørkere er. 

\textcolor{myGreen} {p1:} \textbf{Ja det er litt forskjellige farger (latter). Det her er perioden med depresjon og her traff jeg kjærsten min og så startet jeg terapi her og så 3 år med terapi som jeg egentlig ute her.}

\textcolor{myBlue} {Interviewer:}\textbf{Peker på tegningen, det går nedover til et bunnniva og opp igjen.}

\textcolor{myGreen} {p1:} Over en periode på 3 år.

\textcolor{myBlue} {Interviewer:}Er det noen andre ting dere vurderte å inkludere. 

\textcolor{myGreen} {p1:} Venner

\textcolor{myYellow} {p2:} Ja

\textcolor{myGreen} {p1:} Venner jeg hadde, skolegang, hvilke lærer jeg hadde, hvilke fag jeg hadde, hvem jeg bodde hos.

\textcolor{myBlue} {Interviewer:}Hadde du noen (
\textcolor{myYellow} {p2:}

\textcolor{myYellow} {p2:} Ja jeg tenkte også på venner, de som kom inn igjen og nye venner. Når jeg ble innskrevet i en forening, hva som skjedde der hvordan det påvirket meg. Det skjedde jo veldig mye på en stund. Og jeg fikk være med på veldig mye ting da. Hvordan jeg ble da og hvordan det påførte meg. Litt sånne ting som ikke bare er sykdom men også som ting utenom.

\textcolor{myBlue} {Interviewer:}Ja for det er jo helhetsbilde.

\textcolor{myYellow} {p2:} Ja du ser jo liksom hele mennesket.

\textcolor{myBlue} {Interviewer:}Jeg lurte også på om dere idag dokumenterer noen ting i hverdagens deres. Da tenker jeg for eksempel på treningslogg, blogg, dagbøker, almanakker. Er det noe slikt dere bruker? Eller har brukt i perioder?

\textcolor{myYellow} {p2:} Noen ganger har jeg på kvelden og sånt. Tenker jeg litt over de siste dagene og ukene. Og jeg er veldig glad i musikk, musikk og spille og synge og sånn. Så jeg på en måte skriver det ned og ser om det kan bli en sang ut av det.

\textcolor{myBlue} {Interviewer:}Ja.

\textcolor{myGreen} {p1:} Ja skriver litt noveller og sånn, for et par år siden. Som ikke handlet om meg men handlet om meg. Men det ble aldri noe ut av det. Mange (ganger) har jeg prøvd å skrive om ting men det har jo aldri gått sånn jeg hadde aldri noe system på det. 

\textcolor{myBlue} {Interviewer:}Ja

\textcolor{myGreen} {p1:} Men noe som hadde vært kult er jo å ha en kalender hvor man kunne for eksempel via fargekoding og som ungdom er det jo perioder som, hvor det går ganske lang tid eller kort tid som man har på en måte en farge, der. Hvis du kan passe den, så tar du det ut over en farge så får du.

\textcolor{myBlue} {Interviewer:}Så du får når forskjellige ting skjedde. 

\textcolor{myGreen} {p1:} Ja så du kan liksom se tidsperioder da.

\textcolor{myBlue} {Interviewer:}Har dere på noen tid kunne tenke dere å gå tilbake og få mer innsikt i hva dere gjordt tidligere. Har dere hatt det behovet i løpet av sykdomsbildet?

\textcolor{myYellow} {p2:} Ja jeg fikk jo for eksempel en bok jeg kunne skriv inn for eksempel det jeg gjorde den dagen. Hvilken lege jeg hadde og litt sånne type ting. Og det er mange ganger jeg har gått igjennom og sett på boka hva som skjedde da. Og for litt sånn mimre litt. 

\textcolor{myBlue} {Interviewer:}Følte du at det hjalp.

\textcolor{myYellow} {p2:} Ikke sånn positivt eller negativt men (mimrende) det husker jeg veldig godt og sånn det var sånn det var å.

\textcolor{myBlue} {Interviewer:}Er det noen hovedkategorier dere ville trekke frem, så vi tenkte på sykehusbesøk, nye medisiner, nye behandlinger er det noen sånn større ting dere kunne tenke dere å ha oversikt over over lengre perioder.

\textcolor{myGreen} {p1:} Jeg fikk jo resept på medisiner men jeg brukte det aldri, men hadde jeg brukt de ville det vært veldig interresant.

\textcolor{myBlue} {Interviewer:}Ja

\textcolor{myGreen} {p1:} Og da kunne sette effekten på tidslinjen, hvis det hadde fungert. Jeg brukte medisin for insomnia og du kunne det kanskje hatt en liten effekt på tidslinjen, som mindre stresset eller noe.

\textcolor{myBlue} {Interviewer:}Har du noen tanker.

\textcolor{myYellow} {p2:} Jeg tenker jo akuratt det samme, jeg har jo tatt massevis av medisiner, jeg tenkte også når jeg tok dem at jeg merket stor forskjell så det hadde vært fint å sett at den dagen var jeg sånn når jeg gikk på den og \dots

\textcolor{myBlue} {Interviewer:}Vi har noen ting vi har tenkt på som ville vært naturlig, som å endre lege eller få nye legge, innleggelse på sykehuset, er det noen måter. Du nevnte en dagbok du hadde fått, hvis du hadde forsatt å bruke den.

\textcolor{myGreen} {p1:} Ja jeg brukte den bare på sykehuset.

\textcolor{myBlue} {Interviewer:}Ja hvis du hadde brukt den forsatt ville den hatt nytte for deg.

\textcolor{myGreen} {p1:} Det er en bok for da man er innlagt på sykehuset. Men det hadde kanskje vært fint å ha en bok for etter. Sånn at man kan skrive inn hvordan er nå, det tror jeg hadde vært veldig fint for meg hvertifall, jeg hadde hatt veldig nytte av det.

\textcolor{myBlue} {Interviewer:}Dere hadde jo en fantasifase istad (tidligere på workshop) så hvis dere tenker på dette prosjektet her innenfor fri-fantasi. Er det noe det kunne ha hjulpet dere med da? \dots hvis dere hadde hatt en tidslinje hvordan ville den fungert.

\textcolor{myGreen} {p1:} Jeg tenker at det hadde vært veldig kult med en nettside som bare er en kalender eller en tidslinje men ikke for framtiden men for fortiden. Og du legger inn og kan koble det opp mot sosiale medier som du har egentlig bare masse bilder og ting ifra fortiden tid eller ifra en viss periode og så kan du nyansere ulike år eller deler av livet av ditt og litt tydeligere bilder på hvordan det har gått og hvordan ting har forandret seg i forhold til hvordan du har det nå.

\textcolor{myBlue} {Interviewer:}Sammen med bildene og årene er det noe annet du ville koble. 

\textcolor{myGreen} {p1:} Ja kanskje det, hvis jeg kunne lagt inn novellene og sett da at jeg var så langt så skrev jeg noe der. Eller når jeg hadde det bra så skrev jeg det her om den tiden. Det hadde vært

\textcolor{myBlue} {Interviewer:}Har du noen lignende ideer, hvis der fri fantasi og du har et tidslinjeverktøy som beskriver litt av det vi har snakket om

\textcolor{myYellow} {p2:} Jeg tenker litt sånn som det hele prosjektet er, det var en egen app på det du kunne legge inn egentlig alt som er i hverdagen din og hva som skjedde da og det er egen app på det. 

\textcolor{myBlue} {Interviewer:}Og når du tenker på innenfor den appen er det, det var nevnt bilder er det noen andre ting du kommer på i farten som du ville taste inn på appen.

\textcolor{myGrey}{Note:} ingen svar.

\textcolor{myGreen} {p1:} Lydfiler, egentlig alt av media.

\textcolor{myBlue} {Interviewer:}Dere nevnte jeg kalender og humør, ville det være naturlig å kople opp mot den virkelig kalender. Bruker dere kalenderapper idag.

\textcolor{myYellow} {p2:} Ja jeg bruker det veldig mye.

\textcolor{myGreen} {p1:} Ja.

\textcolor{myBlue} {Interviewer:}Er det da naturlig å koble dette.

\textcolor{myGreen} {p1:} Kalender bruker jeg til framtidige ting. Det ville blitt litt anderledes. Men over lang tid er det jo veldig relevant. 

\textcolor{myBlue} {Interviewer:}Ja

\textcolor{myGrey}{Note:} tid.

\textcolor{myBlue} {Interviewer:}Da må vi desverre avslutte, tusen takk for hjelpen det har vært kjempeinterresant. Ville bare si at de her (bildene) er det bare jeg som skal se på. Hvis de brukes i oppgaven så vil det ikke være mulig å identifisere de med dere i det hele tatt. Dere skal være trygge på det. Tusen takk for hjelpen. 








\end{document}