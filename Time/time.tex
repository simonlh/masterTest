\documentclass[11pt,UKenglish, a4paper]{article}
\usepackage[utf8]{inputenc}
%--fonts--
\usepackage[T1]{fontenc}
\usepackage[bitstream-charter]{mathdesign}

%--packages--
\usepackage[UKenglish]{babel}
\usepackage{csquotes,textcomp,varioref}

%--Sitater i starten av dokumentet--
\usepackage{epigraph}

%--Notater i margen og geometri for størrelse--
%\usepackage{geometry}
%\usepackage{marginnote}

\usepackage{graphicx}
%--color--
\usepackage[dvipsnames]{xcolor}

%Setter inn pdfer
\usepackage[final]{pdfpages}

%-linespace--
\linespread{1.3}

%-Mer advansert liste--
\usepackage{enumitem}

%--hyperlinks-- include 
\usepackage[colorlinks=false, pdfborder={0 0 0}]{hyperref}

%--include subfiles-- 
\usepackage{subfiles}

%--fullpage--
%\usepackage{fullpage}

%--Uio-Front-Page--
%removed until it works \usepackage{ifikompendiumforside}

%--bibliography- -sortlocale=nb_No,
\usepackage[backend=biber, sortcites, defernumbers, style=numeric-comp, maxnames=2, natbib=true, backref, sorting=none, url=false]{biblatex}
%fjernet ifra style=authoryear-icomp
\addbibresource[datatype=bibtex]{Remote.bib}

%farger jeg skal ha med
\definecolor{myY}{RGB}{241, 196, 15}
\definecolor{myB}{RGB}{52, 152, 219}
\definecolor{myG}{RGB}{46, 204, 113}
\definecolor{myLy}{RGB}{149, 165, 166}
\definecolor{myR}{rgb}{231, 76, 60}

%--Author and Title--
\author{Simon Lysne Hyenes}
\title{Time}

%--Latex Optimalization--
\tolerance = 5000
\hbadness = \tolerance
\pretolerance = 2000

%--Starter--
\begin{document}
\section{Time}
In this section I will try to present some of the theoretical and conceptual issues relating to time.

Attempting to extrude knowledge about an issue and it's relation to time will often set off a tangent of inquiries. Robert Levine describes a situation where researchers find: 

	``chains of associations and tangents about so many aspects of time---for example, the time of physics, biology, health, culture, personal relationships, music, art---that it sometimes creates a mass free association about experience itself''\cite[preface~xvii]{Levine2008Geography}


This ``enormity and diffuseness''\cite[preface~xvii]{Levine2008Geography} tempts the researcher to identify a profound and fundamental understanding of the human experience. In this thesis I am considering time from several aspects. Within different aspect I am considering how time is related to visualization, mastery and the ``life-worlds'' of the teenage patients. 
 
 \subsection{Time and Visualization}
Time is an abstract concept that has bewildered philosophers, scientist and poets through the ages.

Some elements of time are universally experienced, for instance our inability to escape it's linear progression and the circadian and natural patterns of our solar system. Boroditsky refers to these conceptual properties as ``universal across cultures and languages''\cite[p.~4]{Boroditsky2000Metaphoric}. These universal properties have been studied in various fields such as psychology, linguistics and biology. From  has lead researchers to consider that human represent time spatially. 

\paragraph*{Spatial Time}

Using metaphoric mapping theory Boroditsky researched and theorized how abstract concepts such as time are made concrete and organized through use of metaphors. Weger and Pratt researched whether time is represented using a spatial metaphor\cite{Weger2008Time}. Weger describes time as a one-dimensional concept that can't be observed. Our difficulty with time may stems from our inability to sense it directly-- instead inferred indirectly through other senses.  

Bonato et. al refers to time-as-space as a ``mental time line''\cite[p.~2258]{Bonato2012When} The ``mental time line'' hypothesis is built on four features. Time is represented spatially along a line. The ``mtl'' is cultural and embodied and primarily seens as conforming to writing and reading directions. The ``mtl'' has relative points, where one event can be aligned according to another but not fixed on a set position itself. Thinking using the ``mtl'' requires ``spatial attention''\cite*[p.~2258]{Bonato2012When}. Bonato review several studies into and related to the ``mtl'' and show support for the hypothesis that time is in many regards represented spatially. 

 \subsection{Phenomenological time}
\epigraph{``Time'' he said, ``is what keeps everything from happening at once''}{\cite[Ray Cummings, p.~46]{Cummings2005Girl}}

As previously mentioned time has certain universal properties. The flow of time is separated and framed by many different means, cyclical such as days and seasons or distinct such as unique events. These properties have been adapted by cultures, peoples and individuals in distinct fashions. The usual example is left-right orientation of time in western culture and the dialectic between event-time and clock time\cite[]{Levine2008Geography}. 

In \textit{A Geography of Time} Robert Levine describes how our relation to and perception of time is affected by culture, environment and personality. Together these factors form a ``pace of life''\cite[]{Levine2008Geography} which is formed not only by individuals but also through the social behaviour of the larger community.

\paragraph*{Time Perspective}
Events, decision and memories are all related to ones relation to time. 
Zimbardo and Boyd expand upon Kurt Lewin's concept of ``time perspective'' defining it as a ``nonconcious personal attitude that each of us hold toward time and the process whereby the continual flow of existence is bundled into time categories that help give order, coherence, and meaning to our lives''\cite[p.~51]{Zimbardo2008Time}. They divide the attitude towards time into six categories: 
\begin{itemize}
\item{Past-negative}
\item{Past-positive}
\item{Present-fatalistic}
\item{Present-hedonistic}
\item{Future}
\item{Transcendental-future}
\end{itemize}\cite[p.~52]{Zimbardo2008Time}
Each category involves a bias and forms the lens through which memories, decisions and values take shape. These categories pinpoint certain positive biases which should be promoted over the more destructive perspectives. Zimbardo and Boyd point to Emmons and McCullough work on gratitude and it's relation to attitudes about the past. Here participants with higher degrees of gratitude had comparatively higher levels of well-beign. Present perspectives are negative when they disregard opportunity cost and over-value close and present conditions over future outcomes. Lastly future time perspectives are related to hope, realism and belief in planning for future results\cite[]{Zimbardo2008Time}. 
Promoting past-positive and a present/future time perspectives is considered by Zimbardo, Boyd and many others to be fruitful for personal development. 

\paragraph*{Temporal Structuring}
The patient group and their relation to time is one of their many relations. Orlikowski and Yates 

Western culture has adopted to a clock time that... 

Taylorism.

Time is personal - p.77

Time is money. - Quantitatively controlled effiency.
Benjamin Franklin has been cited as claiming that ``time is money''

Tardiness
\subsection{Time and Mastery}
 

\end{document}