\documentclass[11pt,UKenglish, a4paper]{article}
\usepackage[utf8]{inputenc}
%--fonts--
\usepackage[T1]{fontenc}
\usepackage[bitstream-charter]{mathdesign}

%--packages--
\usepackage[UKenglish]{babel}
\usepackage{csquotes,textcomp,varioref}

\usepackage{graphicx}
%--color--
\usepackage[dvipsnames]{xcolor}

%Setter inn pdfer
\usepackage[final]{pdfpages}

%-linespace--
\linespread{1.3}

%-Mer advansert liste--
\usepackage{enumitem}

%--hyperlinks-- include 
\usepackage[colorlinks=false, pdfborder={0 0 0}]{hyperref}

%--include subfiles-- 
\usepackage{subfiles}

%--fullpage--
%\usepackage{fullpage}

%--Uio-Front-Page--
%removed until it works \usepackage{ifikompendiumforside}

%--bibliography- -sortlocale=nb_No,
\usepackage[backend=biber, sortcites, defernumbers, style=numeric-comp, maxnames=2, natbib=true, backref, sorting=none, url=false]{biblatex}
%fjernet ifra style=authoryear-icomp
\addbibresource[datatype=bibtex]{Remote.bib}
%\addbibresource[datatype=bibtex]{Kilder.bib}

%farger jeg skal ha med
\definecolor{myY}{RGB}{241, 196, 15}
\definecolor{myB}{RGB}{52, 152, 219}
\definecolor{myG}{RGB}{46, 204, 113}
\definecolor{myLy}{RGB}{149, 165, 166}
\definecolor{myR}{rgb}{231, 76, 60}

%--Author and Title--
\author{Simon Lysne Hyenes}
\title{Methods}

%--Latex Optimalization--
\tolerance = 5000
\hbadness = \tolerance
\pretolerance = 2000
%--Starter--

%--Starter--
\begin{document}
\section{Methods}
\subsection{Workshop 1}
Workshop 1 was conducted together with the whole KULU project. We traveled to AHUS and visited the youth council at Ahus. From Ahus there was the youth council, the youth council coordinator and the head of research(fiks). From the KULU project we were three master students, one phd-candidate as observer and our thesis supervisor Maja van der Velden. For me and the other master-students this workshop was our first meeting with the youth council. The KULU project however has worked with the group in similar capacities before and were well aqcuantied. 
\subsubsection{Recruitment}
We were working the youth council at AHUS\dots. (more here)

The number of participants was limited to the available members of the youth council. There were a total of 8 members available. This worked nicely with the alloted time and size of the rooms. Larger groups would have required either more time or a different set-up for the whole group. 

Within the parameteres 8 participants was plenty and I eventually found myself running low on time. This was mainly due to a raised engagement I had not experienced in my pilot-studies. 

Not answered: What determines the sizes of qualitative workshops?

\subsubsection{Planning}
The workshop were planned together with the rest of the KULU group to ensure that we could divide the group. The planning extended to consider the materials, seating, data-recording, time allotment and how to not uneccessarily move patients with disabilities. Here we used the experiences of the other members to find the most optimal solutions. 
\subsubsection{Consent}
In this research project we are working with sensitive data and young patients. To address legal and ethical issues we collectively used a consent form before the workshop was started. This was done once for the whole group by our master supervisor Maja van der Velden. (om samtykkeskjema)

Before I started the workshop I elaborated upon the purpose and goals of the exercise and informed the participants that we were recording the workshop and that the audio would anonymized through transcription. 

\subsubsection{Organization}
How was the workshop set-up? What level of detail is useful here?
First introduction and food.
Then 3 groups in workshops. 

\subsubsection{Preparation}
For workshop 1 we were working with between 8-10 participants and a total of three hours. With about 1.5 for conducting workshop and the rest for introductions, food and conclusion. Of the 1.5 hour I had 3 consequtive groups for 20 minutes each, with 2, 3 and 3 members. 

\subsubsection{Planning}
I considered several alternatives for the first workshop, mainly methods from participatory design. Herewith I considered card-sorting, future workshop, collage, storyboarding. I wanted to use a visual methods but I also wanted to discuss issues with the participants. In the first stage I planned a short introduction to the thesis topic and the agenda for the workshop. Here I had prepared 12 print-outs of visualizations of timelines. Trying to show a large range of options and not steer the group beforehand. 

For the rest of the workshop I wanted to use participatory techniques. Sanders et. al divide participatory methods into three categories:

\begin{enumerate}[label=\bfseries\arabic*]
\item{Making tangible things}
\item{Talking, tellings and explaining}
\item{Acting, enacting and playing} 
\cite[p.2]{Sanders2010Framework}
\end{enumerate}

Due to the limited time I could only use one of the categories. I decided to attempt participatory sketching combined with semi-structured interviews.

In participatory skethcing the exercise was drawing a timeline freely based on very few criteria. This is within making tangible things, specifically 2-d collages. Sanders et. al. put 2-d collages as one technique that can work with all four purposes---``probe, prime, understand and generate''\cite[p.3]{Sanders2010Framework}. I also prepared questions to probe the participants during the drawing. The aim of the first part was to engage the participants in a deeper mode. Visualization and timelines are hard to phantom and grasp. Therefore having the participants themselves brainstorm and think I wanted to raise their engagement with the topic.

The last part was a semi-structured interview discussing both the produced drawings and also several related issues. I tried to engage the users in enacting by having them explain their drawings and their future use. The aim here was to move beyond the drawings and discuss previous experiences with similar tools and their future visions for what such a tool could entail. 

To test and prepare this setup I conducted 2 short pilot-studies.

Based on the experiences from the pilot-study I made a workshop guide containing text and questions for each parts. I also gathered a broad selection of drawing supplies, rulers, erasers and other nifty tools. 
\subsubsection{Pilot-study}
About the pilot-study.
Problems!

\subsection{Between the workshops}
\subsection{Workshop 2}
\end{document}