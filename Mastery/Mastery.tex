\documentclass[11pt,UKenglish, a4paper]{article}
\usepackage[utf8]{inputenc}
%--fonts--
\usepackage[T1]{fontenc}
\usepackage[bitstream-charter]{mathdesign}

%--packages--
\usepackage[UKenglish]{babel}
\usepackage{csquotes,textcomp,varioref}

\usepackage{graphicx}
%--color--
\usepackage[dvipsnames]{xcolor}

%Setter inn pdfer
\usepackage[final]{pdfpages}

%-linespace--
\linespread{1.3}

%-Mer advansert liste--
\usepackage{enumitem}

%--hyperlinks-- include 
\usepackage[colorlinks=false, pdfborder={0 0 0}]{hyperref}

%--fullpage--
%\usepackage{fullpage}

%--Uio-Front-Page--
%removed until it works \usepackage{ifikompendiumforside}

%--bibliography- -sortlocale=nb_No,
\usepackage[backend=biber, sortcites, defernumbers, style=numeric-comp, maxnames=2, natbib=true, backref, sorting=none, url=false]{biblatex}
%fjernet ifra style=authoryear-icomp
\addbibresource[datatype=bibtex]{Mastery.bib}
\addbibresource[datatype=bibtex]{Remote.bib}

%--Author and Title--
\author{Simon Lysne Hyenes}
\title{Mastery}


%--Latex Optimalization--
\tolerance = 5000
\hbadness = \tolerance
\pretolerance = 2000

%--Starter--
\begin{document}

%\clearpage
%Insert table of contents and then skip to next page
%\tableofcontents
%\clearpage

\section{A theory of mastery}
Younger defines mastery as ``a human response to difficult or stressful circumstances in which competency, control, and dominion have been gained over the experience of stress''\cite[p.76]{Younger1991Theory}. This is not mastery as having mastered a craft or skill but related to one's ability to handle situations. Mastery here relates to a positive outcome from dealing with stressful and difficult situations. 

\subsection{Philosophical and Conceptual elements}
Younger points to stoicism and existentialism as the philosophical and conceptual foundations. Despite their association with futilism they also deal with accepting and confronting reality. as perceived to the fullest of ones abilities. 
Younger points to four properties of mastery. The first deals with a``sense of control..over a situation that created a sense of vunerability and over ones own life''\cite[p.81]{Younger1991Theory}. The second deals with knowledge of how to deal with and avoid similar events. The third deals with regaining confidence and belief in oneself. The fourth dealing with ``having found alternative sources of satisfaction for what is lost''\cite[p.81]{Younger1991Theory}. Youngers relates mastery also to a higher quality of life. This comes from viewing mastery as not only expressed as an interpersonal relation to oneself but also to others. 

\subsection{Conceptual Elements}
\begin{enumerate}[label=\bfseries\arabic*]
\item \textbf{Certainty}\hfill \\
Certainty is a state of negotiation ``predominately free of self-doubt'' and somewhat aligned with other involved parties\cite[p.84]{Younger1991Theory}.
\item \textbf{Change}\hfill \\
Attempt to reduce stressor by problem-solving, decision-making and action\cite[p.84]{Younger1991Theory}. Younger relates change to our ability to affect fate. As with fate change is the whole-hearted attempt and change may be outside the scope of possibilities.
\item \textbf{Acceptance}\hfill \\
Dealing with and coming to terms with the actual circumstances at hand. Reinvest, change or find alternate sources are the main tools for mastering acceptance. Acceptance is the stoic realization of the actual possibilities but also an attempt to move on by reinvesting in new goals or relations. %From Camus and existentialism acceptance allows us to create new meaning and take solace.
\item \textbf{Growth}\hfill \\
The last state relates to a positive outcome of the situation through new skills, experience, alternate goals, regained or raised strength. Growth in mastery means a new view of similar events through experience and knowledge. Younger points to this mastery as building a healthier community by a raised ``compassion offer others in similar situations''\cite[p.86]{Younger1991Theory}.
\end{enumerate}

These four properties are intertwined, this involves a feedback loop where states may build or destruct other states. Certainty is the conditions and growth the product of mastery. Acceptance and change however may be viewed intertwined elements. Acceptance deals with coming to terms with impossible and change with the possible. Younger points to that change then may lead to ``new knowledge and skills'' and acceptance to ``investment of life energies in new people or new pursuits''\cite[p.87]{Younger1991Theory}. Both of the results allow for change to a state of mastery. 

\section{Timelines and mastery}

Younger's four properties points to several areas the timeline should strive to support. Mastery depends upon several different stages that cannot be reduced to prescriped actions but depend wholly on the patients intrapersonal mode, experiences and circumstances.

This support may be considered as either happening during a process of mastery--- or after as a narrative or documentation of a process of mastery. This means supporting different temporal modalities, types of data and levels of use and engagement.   

The four main properties certainty, change, acceptance and growth all point to different areas the timeline can support. 

Mastery depends upon certainty and change, this can be supported through examplars of earlier actions. Mastery means constructing ones own story and future reality through acceptance and growth. Acceptance and growth are interpersonal modes that highlight the need to not only focus on past event and actions. Having a focus towards future goals, alternate pursuits outside of the `stressors' may help appectance. Growth may be a state where such as the timeline may have less perceived utility for the patients. It however may still work in the two fashions---as a reminder and a narrative of the process that lead to a state of growth and further as documentation of the altered state that growth lead to. If succesful this awareness further supports the raised compassion toward the community and others. 

The focus on mastery in contrast with other similar concepts such as autonomy is useful for several reasons. Supporting peoples autonomy means raising or giving people the ability to affect areas that inflict them. Focusing on mastery would mean supporting autonomy--- not as a primary action but as result from self-inflicted change, acceptance and growth.
\end{document}
