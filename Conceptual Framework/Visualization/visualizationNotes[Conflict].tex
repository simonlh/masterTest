Tufte: images as narratives- ``the idea is to make design that enhance the richness, complexity, resolution, dimensionality, and clarity of the content''\cite[p.~10]{Tufte1997Visual}. 

Envisioning - visual strategies for design: color, layering, and interaction effects

Comparison.

Visualization strenght comes from comparison. There are many design principles that can help make comparison easier and more truthful. Lidwell et. al point to three main ideas: apples-to-apples, single context, and benchmarks\cite[p.~43]{Lidwell2010Universal}. The first idea is to only design for comparison when the variables are comparable in a fair manner. The idea is to use common measures and units and to account for differences. 

Single context, comparison work best within a single contextual frame, i.e one graph with two variables instead of two graphs with one variable. 


Benchmarks - a standard value for comparing other values and results\cite[p.~43]{Lidwell2010Universal}. 

\cite[p]{}
``The general principles of how to assemble and present data visually, in the most efficient and pleasing manner''

``Micro-reading'' individual stories about the data. 

Tufte belongs to those who assert that there is a false equation of simpleness and clarity in information design. Simpleness is instead viewed as a preference and an aesthetic quality. Tufte describes his alternative solution as ``What we seek instead is a rich texture of data, a comparative context, an understanding of complexity revealed with an economy of means''\cite[p.~51]{Tufte1990Envisioning}. Instead working with complexity, contradiction, and intricacy are deemed as more descriptive of the ``worlds we seek to understand''\cite[p.~51]{Tufte1990Envisioning}. 

``''