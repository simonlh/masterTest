\documentclass[11pt,UKenglish, a4paper]{article}
\usepackage[utf8]{inputenc}
%--fonts--
\usepackage[T1]{fontenc}
\usepackage[bitstream-charter]{mathdesign}

%--packages--
\usepackage[UKenglish]{babel}
\usepackage{csquotes,textcomp,varioref}

\usepackage{graphicx}
%--color--
\usepackage[dvipsnames]{xcolor}

%Setter inn pdfer
\usepackage[final]{pdfpages}

%-linespace--
\linespread{1.3}

%--hyperlinks-- include 
\usepackage[colorlinks=false, pdfborder={0 0 0}]{hyperref}

%--fullpage--
%\usepackage{fullpage}

%--Uio-Front-Page--
%removed until it works \usepackage{ifikompendiumforside}

%--bibliography- -sortlocale=nb_No,
\usepackage[backend=biber, sortcites, defernumbers, style=numeric-comp, maxnames=2, natbib=true, backref, sorting=none, url=false]{biblatex}
%fjernet ifra style=authoryear-icomp

%farger jeg skal ha med
\definecolor{myY}{RGB}{241, 196, 15}
\definecolor{myB}{RGB}{52, 152, 219}
\definecolor{myG}{RGB}{46, 204, 113}
\definecolor{myLy}{RGB}{149, 165, 166}
\definecolor{myR}{rgb}{231, 76, 60}

%--Author and Title--
\author{Simon Lysne Hyenes}
\title{Transkribsjon av workshop 1: Gruppe 2}


%--Latex Optimalization--
\tolerance = 5000
\hbadness = \tolerance
\pretolerance = 2000

%---------start------------
\begin{document}
\maketitle{}
\section{Introduksjon}
Gruppe 2 hadde tre deltakere. De er markert med tre forskjellige farger og navnet p1, p2 og p3. ``P'' står for participant. 
\section{Gruppe 2}
i: Jeg heter Simon forresten bare sånn at det er sagt. Oppgaven her handler om tidslinjer og om det går ann å bruke tidslinjer som et verktøy i hverdagens deres til å få oversikt over ting som skjer over korte perioder og også over lengre perioder, måneder eller år. Siden det er KULU prosjektet så tenker vi relatert til helse og dere som pasienter, men også litt friere og. Så først skal vi starte med \dots om dere vet hva tidslinjer er? Hva jeg prater om. 
p1: Ja om det sånn tid i perioder som vi har hatt på skolen. (imiterende) \textit{Her startet hele prossesen før annen verdenskrig}. Så tegner man en sånn prosess hvor ting skjer hvert år
i:Ja
p1: Så skal det være eksempel over dager eller
i: Absolutt
p2: Jeg tenkte mer på
P3: Jeg tenker mer på streker
i: Jeg har med noen eksempler. Vet du. Vi har jo sånn klassiske (viser tidslinje med historiske gjenstander) hvor det er massevis av ting over lengre perioder.
p3: Bra tidslinje ville jeg si.
i: Og så har vi en veldig gammel hvor det bare var tekst og..
p2: Fy søren (imponert)
p3: Sånn vil vi ikke ha (spøkende)
alle: (latter)
i: Så er det den klassiske historien da, 
p2: Musikkhistorien
p1: Nei (?)
i: Her er et bilde fra en nettside hvor tidslinjen beveger seg innover. Så du scroller opp.
p1: Kult
p2: Fancy
p3: Dritkult
i: Så er det da Facebook, som jeg regner med at dere er kjent med. Hjemmesiden er jo også en tidslinje. Bare at den er horisontal ikke. Så\dots vi har noe flere eksempler her vi skal gå kjapt gjennom de. (Viser bilder ifra Jawbone Up appen med aktivitet, søvn og trening) Det er her er min personlige\dots eller en treningsapp jeg bruker som viser oversikt over aktivitet i løpet av en dag. 
p3: Oi. Hva heter den appen der.
i: Det er en jawbone up.
p1: Det minner meg litt om\dots
p3: (kjapt)nike
p1: fuelband ja
i: Ja det er en sånn (intervjuer viser at han har på jawbone up24 båndet). Det er bare for å vise
p3: sånn (\textcolor{myLy}{umulig å høre}). 
alle: (latter)
p3: VI har hatt det alle vi gutta i klassen.
i: Ja
p3: (spørrende) Hvem sover mest?
p1: Det kunne ikke\dots
p3: Ja, (fleipende) det er ikke meg
p1: Nei!
i: Skal vi ta de fire siste her. Det her er en sosiale medier oversikt (nettsted) med bilder, hendelser og steder over tid.
p3: Kult
p2: Kult
i: Så her er det bare for å vise (viser sirkeltidslinje) at det ikke trenger å være en linje det kan være en sirkel eller andre ting. \dots. Så det to siste her er personlige, det er ikke mine, men den ene er en historie ifra en person som forteller om hvordan hun ble en lærer (viser skriblet tidslinje) over tid. Den andre er en oversikt over følelser og hendelser i livet til (viser ``Amelia'' sin tidslinje). Her er positive ting (peker øverst) og negative ting (peker nederst) fordelt over.
p3: Den var stilig.
p1: Ja, litt rotete men stilig.
i: Ja så tankegangen var at vi tenkte å la dere starte med å få tegne. For å beskrive litt av deres, det er viktig her at dere\dots det er ikke noen riktig måte å gjøre det her på. Det er ikke noen feil måte å gjøre det på heller og det trenger ikke være pent. Det trenger ikke tegne rette streker, eller bruke farger, dere kan rote det til. Bare at dere beskriver for dere. Så er det da, det kan jo være, det kan jo gjøre hva dere vil, men det er lettere hvis dere tenker på at det skal være et verktøy for andre pasienter eller for pasienter som starter å bruke verktøyet skulle ha en lengre tidsoversikt over hele pasientforløpet. 
p1: Hmmm
i: Men skal vi prøve å starte. Det er jo massevis av tegneutstyr her. Det er bare å spørre så kan jeg hjelpe dere. 
p3: Ja
p1: \dots
p3: (fleipende) Jeg er jo knakende god til å tegne
i: Vi har linjaler, som er bøyelige\dots
p2: Den var dritkul
p3: Kult
i: Vi har svære tusjer, vi har mindre, vanlige penner og fargeblyanter. Så hvis dere ikke vet hvordan dere skal starte er det bare å sette et punkt på arket et eller annet sted. Og så er det egentlig bare å. 
merknad:(gruppen ser undrende ut)
i: Ingen av dere er klare eller ingen av dere har noen\dots
p3: Nei, jo, neiiii 
p1: Nei
p2: Ikke noe akuratt\dots
p3: Ikke der (klapper)
i: Da kan jo vi starte med noen eksempler, det her er ikke ment å være riktig men bare noen ting vi har tenkt på. Som for eksempel legebesøk, over tid. Man kan ha oversikt over sykdom, som dagsform eller innleggelse. Oversikt over energi, humør, stress i hverdagen\dots medisinbruk, nye leger, sykehusopphold. Det her var bare noe, det trenger ikke være det
p3: Ikke bare sjukehus det er snakk om
i: Nei det trenger ikke bare være det, det kan også være om hvordan dere personlig har det eller større hendelser i livet deres eller andre ting. 
p3: Ja
i: Så det første å starte med, er jo om dere ønsker å ta et år, ønsker dere å ta flere år. Ønsker dere å ta en dag.
p3: Ja, jeg hadde tenkt en måned jeg ass\dots
i: Ja
p3: Bare sånn
i: Absolutt
p2: Jeg tenkte et halvt år
p3: Jeg har brukt det mer som en møteplanlegger
p2: (ler)
p3: (spørrende)Hva gjorde jeg i år
i: Vil dere ha noen penner eller 
p3: (tok en penn) Ja rosa er fint vet du.
p2: Blir mye rosa idag
p3: (spøkende) Er glad i rosa. (enda mer spøkende) Var yndlingsfargen min da jeg var liten
p2: Seriøst (deadpan)
p3: Ja. Ikke min stolteste innrømmelse\dots men
p2: Det er jo lov med litt
p1: Jeg tror ikke jeg kommer på med noen deler.

\end{document}


