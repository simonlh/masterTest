1. Focus towards future users
2. From the lab to the field
3. What is the purpose - actively involve people design for and the stakeholders.

At it's

Simonsen and Robertson point to two main focused of pd-research. The first is to allow future use of an artifact a chance to contribute in it's design. The second focus is for mutual learning between designers and non-designers access and ability to "envisage future technologies"\cite[p.]{Simonsen2012Routledge}.

5. Inkluder noe om at PD gjelder andre felt en system development.

6. Its core: Interpretive - Ethnographic - Tacit knowledge can be examined ethically and efficiently through cooperation and partnership with the participants. 

7. When to use PD - early in the design process - Visions. 

8. Qualitative methods used - 

9. Why PD - 
	To explore the life-worlds of the participants. To find their own needs and desires. 

	I needed them to help me. Not only understand their needs. 


Ethics. 

Lowgren Stolterman.

10. Tre fokuser - individual project, company arena, national arena - 169. 

Kensing and Blomberg point to three main issues within pd-research. 
The politics of design.
The nature of participation
Methods, tools and techniques. 


``an accountability of design to the worlds it creates and the lives of those who inhabit them''
Schøn:
The  nature draws upon Donald Schön \textit{reflection-in-action} and \textit{reflection-on-action}

Etikk:
From a relation aspect of design 

``VIKTIG'':
Bratteteig and Stolterman (1997) description of visionary designers and the view that the “purpose of the design is not just the designed artefact itself, but changes in the range of possibilities for action in the social organization which will use the artefact” (p.4).